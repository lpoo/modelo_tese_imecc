\chapter{T\'itulo 2}
Esse \'e o segundo cap\'itulo da sua ...

\section{Se\c c\~ao 2.1}
Essa é uma seção da sua tese.

É possível inserir figuras na sua tese. \\
\includegraphics{figuras/unicamp-logo.jpg}

É recomendado deixar as figuras\index{figura} como objetos flutuantes.
\begin{figure}[!htb]
    \center
    \includegraphics[width=4cm]{figuras/unicamp-logo.jpg}
    \caption{Logo da UNICAMP.}
    \label{fig:log_unicamp}
\end{figure}

Figuras também podem ter subfiguras, usando o pacote \texttt{subcaption}, que 
podem ser citadas individualmente: figura~\ref{fig:caption1}.
\begin{figure}[!htb]
    \centering
    \subcaptionbox{\emph{Caption} 1.\label{fig:caption1}}
        {\includegraphics[width=0.45\textwidth]{figuras/unicamp-logo.jpg}}
    \qquad
    \subcaptionbox{\emph{Caption} 1.\label{fig:caption2}}
        {\includegraphics[width=0.45\textwidth]{figuras/unicamp-logo.jpg}}
    \caption{Figura com subfiguras.}
    \label{fig:subfig}
\end{figure}

Além de inserir figuras é possível ``desenhar''\index{figura!desenhar} figuras utilizando o pacote
TikZ\index{figura!TikZ}, como na Figura~\ref{fig:exem_tikz}.
\begin{figure}[!htb]
\centering
\begin{tikzpicture}
\draw (0,0) -- (0,2) -- (2,2) -- (2,0) -- (0,0);
\end{tikzpicture}
\caption{Exemplo do uso do TikZ.}
\label{fig:exem_tikz}
\end{figure}

\section{Seção 2.2}
Além de figuras também é possível inserir tabelas. \\
\begin{tabular}{|c|c|}
\hline
Tabela & Tabela \\ \hline
Tabela & Tabela \\ \hline
\end{tabular}

Assim como as figuras, é recomendado que as tabelas sejam incluidas como
objetos flutuantes.
\begin{table}[!htb]
\caption{Exemplo de Tabela.}
\label{tab:exem}
\centering
\begin{tabular}{c|c}
Tabela & Tabela \\ \hline
Tabela & Tabela \\
\end{tabular}
\end{table}

Além de inserir tabelas manualmente, como na Tabela~\ref{tab:exem}, também é possível utilizar arquivos
\emph{.csv} para criar tabelas. Neste caso, é necessário incluir o pacote
\texttt{csvsimple} que não foi incluído por questão de compatibilidade com
distribuições do \LaTeX \ anteriores a 2009.
