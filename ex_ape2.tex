\chapter{Git}
Git é o atual \textit{state-of-the-art} sistema de controle de versão e tem sido
utilizado em vários projetos, muitos na área de desenvolvimento de software,
sendo o grande destaque o Kernel Linux. Este modelo de tese possui suas versões
controladas por meio do Git e nas próximas sessões será apresentado algumas
dicas de como utilizar o Git para aproveitar melhor este modelo.

\section{Baixando o modelo}
Para baixar o modelo utilizando o Git você deve executar o comando:
\begin{lstlisting}[escapechar=@]
$ git clone https://github.com/r@-@gaia@-@cs/modelo_tese_imecc.git
\end{lstlisting}
Será criado uma pasta \lstinline+modelo_tese_imecc+ com os arquivos do
modelo.

Além da versão oficial do modelo existem algumas outras e para ver uma lista
destas versões você deve executrar o coamndo:
\begin{lstlisting}
$ git branch
\end{lstlisting}
Para selecionar a versões a ser utilizada, execute o comando:
\begin{lstlisting}
$ git branch versao_desejada
\end{lstlisting}
A versão oficial é chamada de \lstinline+master+.

Antes de começar a editar os arquivos, execute os comandos:
\begin{lstlisting}
$ git branch meu_trabalho
$ git checkout meu_trabalho
\end{lstlisting}

\section{Editando o modelo}
Evite ao máximo modificar os arquivos originais do modelo pois isso facilitará a
atualização do mesmo como será apresentado na seção seguinte.

Para adicionar novos arquivos ao controle de versão, execute o comando:
\begin{lstlisting}
$ git add novo_arquivo
\end{lstlisting}
E para adicionar modificações nos arquivos previamente adicionados ao controle
de versão, execute o comando:
\begin{lstlisting}[escapechar=@]
$ git add @-@u
\end{lstlisting}

Para criar uma nova versão da sua dissertação/tese, execute o comando:
\begin{lstlisting}[escapechar=@]
$ git commit @-@m 'Breve descricao do que foi feito ate agora.'
\end{lstlisting}

\section{Atualizando o modelo}
Como este modelo é um trabalho em progresso e não existe nenhuma garantia de que
as deliberações da Comissão Central de Pós-Graduação serão mantidas até que você
termine seu mestrado/doutorado é importante existir uma maneira de você
atualizar sua dissertação/tese já em processo de escrita com as novas
deliberações.

Antes de você atualizar o modelo, execute o seguinte comando:
\begin{lstlisting}[escapechar=@]
$ git commit @-@m 'Preparacao para atualizacao do modelo.'
\end{lstlisting}
Para atualizar o modelo, execute os comandos:
\begin{lstlisting}
$ git fetch
$ git checkout versao_desejada
$ git merge origin/versao_desejada
$ git checkout meu_trabalho
$ git rebase meu_trabalho versao_desejada
\end{lstlisting}
