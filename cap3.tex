\chapter{Título 3}
Esse é o terceiro capítulo da sua tese.

\section{Seção 3.1}
Para facilitar a leitura, é bom incluir o algoritmo de forma adequada e não
como uma lista.
\begin{algorithmic}[2]
  \STATE $i \leftarrow 0$
  \STATE $y \leftarrow 0$
  \FOR{$i \leq 10$}
    \STATE $y \leftarrow y + i$
    \STATE $i \leftarrow i + 1$
  \ENDFOR
\end{algorithmic}

É possível nomear os algoritmos\index{algoritmo}.
\begin{algorithm}
  \caption{Loop infinito.}
  \label{alg:loop_inf}
  \begin{algorithmic}
    \REQUIRE $n \geq 0$
    \ENSURE $y = 1 + 2 + \ldots + n$
    \STATE $y \leftarrow 0$
    \STATE $i \leftarrow 0$
    \IF{$n < 0$}
      \PRINT Entrada inadequada.
    \ELSE
      \WHILE{$i \neq n$}
        \STATE $y \leftarrow y + i$
      \ENDWHILE
    \ENDIF
  \end{algorithmic}
\end{algorithm}

Para algoritmos, recomenda-se utilizar o pacote
\emph{algorithmic}\index{algoritmo!algorithmic@\emph{algorithmic}} pois este
permite a traduçãoo dos ``comandos''.

Em alguns casos, recomenda-se incluir um trecho de
código\index{codigo@código}.
\lstinputlisting[firstline=5, lastline=5, nolol=true]{src/exem.c}

Assim como os algoritmos também é possível nomear os códigos.
\lstinputlisting[firstline=10, lastline=12, caption=Loop em C,
language=C,  frame=single]{src/exem.c}

Para trecho de códigos,  recomenda-se utilizar o pacote
\emph{listings}\index{codigo@código!listings@\emph{listings}} pois
este possui várias opções.
