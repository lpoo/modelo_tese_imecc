% FIXME Ao final, deixe descomentada a linha correspondente ao numero de paginas
% que a sua defesa possui.
\documentclass[12pt, letterpaper, oneside]{book}  % Para menos de 100 paginas.
% \documentclass[12pt, letterpaper, twoside]{book}  % Para mais de 100 paginas.

%%%%%%%%%%%%%%%%%%%%%%%%%%%% Configuração: pacotes %%%%%%%%%%%%%%%%%%%%%%%%%%%%
%%%%%%%%%%%%%%%%%%%%%%%%%%%%%% Pacotes: básicos %%%%%%%%%%%%%%%%%%%%%%%%%%%%%% 
% FIXME Selecione a codificação adequada.
\usepackage[utf8]{inputenc}
% \usepackage[latin1]{inputenc}
\usepackage[T1]{fontenc}
%%TODO Usar fullpage ou geometry?
\usepackage[cm]{fullpage}
%\usepackage[top=3cm, bottom=3cm, left=2cm, right=2cm]{geometry}
\usepackage[brazilian]{babel}
%%TODO Usar espaço na primeira linha?
\usepackage{indentfirst}
%%%%%%%%%%%%%%%%%%%%%%%%%%%%%%% Pacotes: links %%%%%%%%%%%%%%%%%%%%%%%%%%%%%%%
\usepackage[hyphens]{url}
\usepackage{hyperref}
\usepackage{breakurl}
%%%%%%%%%%%%%%%%%%%%%%%%%%%%%%%% Pacotes: ams %%%%%%%%%%%%%%%%%%%%%%%%%%%%%%%% 
\usepackage{amsmath}
\usepackage{amsfonts}
\usepackage{amssymb}
\usepackage{amsthm}
%%%%%%%%%%%%%%%%%%%%%%%%%%%%%% Pacotes: tabelas %%%%%%%%%%%%%%%%%%%%%%%%%%%%%%
\usepackage{multicol}
\usepackage{multirow}
\usepackage{array}
\usepackage{csvsimple}
%%%%%%%%%%%%%%%%%%%%%%%%%%%%%% Pacotes: figuras %%%%%%%%%%%%%%%%%%%%%%%%%%%%%% 
\usepackage{pdfpages}
\usepackage{graphicx}
\usepackage{tikz}
\usepackage{wrapfig}
%%%%%%%%%%%%%%%%%%%%%%%%%%%%% Pacotes: algoritmos %%%%%%%%%%%%%%%%%%%%%%%%%%%%% 
\usepackage{algorithmic}
\usepackage[chapter]{algorithm}
\floatname{algorithm}{Algoritmo}
\renewcommand{\listalgorithmname}{Lista de Algoritmos}
%%%%%%%%%%%%%%%%%%%%%%%%%%%%%% Pacotes: códigos %%%%%%%%%%%%%%%%%%%%%%%%%%%%%% 
\usepackage{textcomp}
\usepackage{listings}
\renewcommand\lstlistingname{C\'{o}digo}
\renewcommand\lstlistlistingname{Lista de C\'{o}digos}
%%%%%%%%%%%%%%%%%%%%%%%%%%%%%%% Pacotes: index %%%%%%%%%%%%%%%%%%%%%%%%%%%%%%% 
\usepackage{makeidx}

%%%%%%%%%%%%%%%%%%%%%%%%%% Configurações: numeração %%%%%%%%%%%%%%%%%%%%%%%%%% 
\numberwithin{equation}{section}
\numberwithin{section}{chapter}
%%%%%%%%%%%%%%%%%%%%%%%%%%%%% Configurações: ams %%%%%%%%%%%%%%%%%%%%%%%%%%%%% 
\theoremstyle{definition}
\newtheorem{thm}{Teorema}[section]
\newtheorem{con}[thm]{Conjectura}
\newtheorem{cor}[thm]{Corol\'ario}
\newtheorem{dfn}[thm]{Defini\c c\~ao}
\newtheorem{exm}[thm]{Exemplo}
\newtheorem{lem}[thm]{Lema}
\newtheorem{obs}[thm]{Observa\c c\~ao}
\newtheorem{pps}[thm]{Proposi\c c\~ao}
%%%%%%%%%%%%%%%%%%%%%%%%%%%%% Configurações: index %%%%%%%%%%%%%%%%%%%%%%%%%%%%% 
\makeindex  % Necessário para gerar o index.
% TODO Inserir configurações adicionais aqui.
  % Arquivo com os pacotes.

%%%%%%%%%%%%%%%%%%%%%%%%% Configuração: dados pessoais %%%%%%%%%%%%%%%%%%%%%%%%%
% FIXME Substituir 'Nome completo do aluno' pelo seu nome.
\newcommand{\autor}{Nome completo do aluno}
% FIXME Se for do sexo feminino, descomente a linha a seguir.
% \def\femaleAuthor{}

% FIXME Substituir 'Título da defesa' pelo título da defesa.
\newcommand{\titulo}{Título da defesa}
% FIXME Se estiver no programa de mestrado, descomente a linha a seguir.
% \def\mestrado{}
% FIXME Deixe descomente apenas a linha referente ao departamento.
% \def\matematica{}
\def\aplicada{}
% \def\estatistica{}

% FIXME Substituir 'Nome completo do orientador' pelo nome completo do seu
% orientador.
\newcommand{\orientador}{Nome completo do orientador}
% FIXME Se for orientado por uma mulher, descomente a linha a seguir.
% \def\femaleOrientador{}

% FIXME Substituir 'Nome completo do coorientador' pelo nome completo do seu
% coorientador. Caso não tenha coorientador, comente a linha a seguir.
\newcommand{\coorientador}{Nome completo do coorientador}
% FIXME Se for coorientado por uma mulher, descomente a linha a seguir.
% \def\femaleCoorientador{}

% FIXME Substituir 'Ano' pelo ano em que ocorreu sua defesa.
\newcommand{\ano}{Ano}

%%%%%%%%%%%%%%%%%%%%%%%%%%% Configuração: definições %%%%%%%%%%%%%%%%%%%%%%%%%%%
%%%%%%%%%%%%%%%%%%%%%%%%%% Configurações: numeração %%%%%%%%%%%%%%%%%%%%%%%%%% 
\numberwithin{equation}{section}
\numberwithin{section}{chapter}


%%%%%%%%%%%%%%%%%%%%%%%%%%%%% Configurações: ams %%%%%%%%%%%%%%%%%%%%%%%%%%%%% 
\theoremstyle{definition}
\newtheorem{thm}{Teorema}[section]
\newtheorem{con}[thm]{Conjectura}
\newtheorem{cor}[thm]{Corol\'ario}
\newtheorem{dfn}[thm]{Defini\c c\~ao}
\newtheorem{exm}[thm]{Exemplo}
\newtheorem{lem}[thm]{Lema}
\newtheorem{obs}[thm]{Observa\c c\~ao}
\newtheorem{pps}[thm]{Proposi\c c\~ao}


% TODO Inserir configurações adicionais aqui.
  % Arquivo com algumas configurações.

%%%%%%%%%%%%%%%%%%%%%%%%%% Início do texto da defesa %%%%%%%%%%%%%%%%%%%%%%%%%%%
\begin{document}
% WARNING Todas as paginas deverão ser numeradas.
%
% As paginas iniciais são numeradas com algoritmos romanos em sua forma
% minuscula.
\frontmatter
%
\vspace*{2cm}
\begin{center}
    % O tamanho da fonte deve ser 16pt.
    % Deve-se utilizar caixa alta.
    {\Large \scshape \autor}
\end{center}
\vspace{3cm}
\begin{center}
    % O tamanho da fonte deve ser 16pt em negrito.
    % Deve-se utilizar caixa alta.
    {\Large \scshape \bfseries \titulo}
\end{center}
\vspace{8cm}
\begin{center}
    % O tamanho da fonte deve ser 12pt em negrito.
    % Deve-se utilizar caixa alta.
    {\scshape \bfseries CAMPINAS \\ \ano}
\end{center}
  % Não edite esse arquivo.
\newpage\mbox{}\thispagestyle{plain}\newpage  % Pagina em branco.
%
% WARNING A folha de rosto precisa ser assinada pelos orientadores.
% FIXME Substitua arquivo folha-de-rosto.pdf por uma copia escaneada, comente
% esta linha e descomente a próxima.
\thispagestyle{plain}
\pagenumbering{roman}
\setcounter{page}{1}

\begin{center}
    {\Large {\sc Universidade Estadual de Campinas} }
\end{center}

\begin{center}
    {\Large {\sc Instituto de Matem\'atica, Estat\'istica e Computa\c c\~ao
    Cient\'ifica} }
\end{center}

\vspace{2cm}

\begin{center}
    {\huge {\sc \autor} }
\end{center}

\vspace{2cm}

\begin{center}
    {\huge {\sc \titulo} }
\end{center}

\vspace{1cm}

\begin{flushright}
    \begin{minipage}[c]{7.5cm}
        {\large
        % FIXME Adeque dependendo do nível.
        % Disserta\c c\~ao de mestrado % descomente de for mestrado.
        Tese de doutorado % descomente de for doutorado.
        apresentada ao Instituto de Matem\'atica, Estat\'istica e
        Computa\c c\~ao Cient\'ifica da Universidade Estadual de Campinas para
        obten\c c\~ao do t\'itulo de
        % FIXME Adeque dependendo do nível.
        % mestre/a % descomente de for mestrado.
        doutor % descomente de for doutorado.
        em
        % FIXME Adeque dependendo da área.
        % Matem\'atica.
        Matem\'atica Aplicada.
        % Estat\'istica.
        }
    \end{minipage}
\end{flushright}

\vspace{1cm}

\begin{center}
    {\large {\sc
    Orientador: Prof. Dr. \orientador

    Co-orientador: Prof. Dr. \coorientador
    } }
\end{center}

\vspace{1cm}

\begin{center}
    \begin{minipage}[c]{0.8\textwidth}
        {\large
        Este exemplar corresponde \`a vers\~ao final da 
        % FIXME Adeque dependendo do nível.
        % tese % descomente se for doutorado.
        disserta\c c\~ao % descomente se for mestrado.
        defendida pelo aluno \autor e orientada pelo Prof. Dr. \orientador
        }
    \end{minipage}
\end{center}

\vspace{1cm}

\begin{center}
    \begin{minipage}[c]{8.5cm}
        \begin{center}
            {\sc
            \rule[1pt]{8.5cm}{.5pt} % Assinatura do orientador

            Prof. Dr. \orientador
            }
        \end{center}
    \end{minipage}
    \hspace{.5cm}
    \begin{minipage}[c]{8.5cm}
        \begin{center}
            {\sc
            \rule[1pt]{8.5cm}{.5pt} % Assinatura do co-orientador

            Prof. Dr. \coorientador
            }
        \end{center}
    \end{minipage}
\end{center}

\vspace{1cm}

\begin{center}
    {\Large {\sc Campinas \\ \ano} }
\end{center}

% \includepdf{folha-de-rosto.pdf}
%
% WARNING A ficha catalográfica deve estar no verso da folha de rosto.
% FIXME O arquivo ficha-catalografica.pdf deve ser sobrescrito com uma cópia
% do arquivo pdf que a biblioteca lhe enviar.
\includepdf{ficha-catalografica}
%
% WARNING A folha de aprovação deve ser assinada pelos membros da banca apos a
% defesa.
% FIXME Substitua o arquivo folha-de-aprovacao.pdf por uma copia escaneada.
\includepdf{folha-de-aprovacao}
%
% FIXME Se não for incluir a dedicatória, comentar a linha abaixo.
\chapter*{}
\begin{center}
    \emph{
    % FIXME Remover as duas linhas abaixo.
    Aos meus \ldots. \\
    Para \ldots.} 
    % TODO Inserir a dedicatória aqui.
\end{center}

%
% FIXME Se não for incluir a epigrafe, comentar a linha abaixo.
%%TODO Escrever o modelo da epigrafe.
\chapter*{}
\vfill
\begin{flushright}
  Epígrafe \\
  Epígrafe \\
  Epígrafe \\
  Epígrafe
\end{flushright}

%
% FIXME Se não for incluir os agradecimentos, comentar a linha abaixo.
\chapter*{Agradecimentos}
% FIXME Escrever os agradecimentos.
Aqui deve-se agradecer quem merece agradecimento

Se recebeu bolsa de algum \'org\~ao de fomento, n\~ao esque\c{c}a de agradec\^{e}-lo.

%
\chapter{Resumo}
\addcontentsline{toc}{chapter}{Resumo}
% FIXME Remover as 3 linhas abaixo.
resumo resumo resumo
resumo resumo resumo
resumo resumo resumo
% WARNING O resumo deve conter no máximo 500 palavras.
% TODO Inserir o resumo em portugês aqui.

\vspace{.5cm}
\textbf{Palavras-chave}:
% FIXME Remover a linha abaixo.
palavra01, palavra02, palavra03.
% TODO Inserir as palavras-chave aqui.

\chapter{Abstract}
% FIXME Remover as 3 linhas abaixo.
abstract abstract abstract
abstract abstract abstract
abstract abstract abstract
% WARNING O abstract deve conter no máximo 500 palavras.
% TODO Inserir o resumo em inglês aqui.

\vspace{.5cm}
\textbf{Keywords}:
% FIXME Remover a linha abaixo.
keyword01, keyword02, keyword03.
% TODO Inserir as palavras-chave em inglês aqui.

\chapter{Abstract}
% FIXME Remover as 3 linhas abaixo.
abstract abstract abstract
abstract abstract abstract
abstract abstract abstract
% WARNING O abstract deve conter no máximo 500 palavras.
% TODO Inserir o resumo em inglês aqui.

\vspace{.5cm}
\textbf{Keywords}:
% FIXME Remover a linha abaixo.
keyword01, keyword02, keyword03.
% TODO Inserir as palavras-chave em inglês aqui.

%
\tableofcontents
% FIXME Comentar as linhas abaixo de não desejar listar as figuras
% apresentadas.
\listoffigures
\addcontentsline{toc}{chapter}{Lista de Figuras}
%
% FIXME Comentar as linhas abaixo de não desejar listar as tabelas
% apresentadas.
\listoftables
\addcontentsline{toc}{chapter}{Lista de Tabelas}
%
% FIXME Comentar as linhas abaixo de não desejar listar os
% algoritmos apresentados.
\listofalgorithms
\addcontentsline{toc}{chapter}{Lista de Algoritmos}
%
% FIXME Comentar as linhas abaixo de não desejar listar os trechos de código
% apresentados.
\lstlistoflistings
\addcontentsline{toc}{chapter}{Lista de Códigos}
%
% FIXME Comentar a linha abaixo de não for apresentar as
% abreviações utilizadas.
\chapter*{Lista de Abreviaturas e Siglas\markboth{Lista de Abreviaturas e Siglas}{}}  % \markboth{}{} é utilizado para corrigir o cabeçalho.

\begin{description}
  % FIXME Remover as duas abreviações/siglas abaixo e incluir as que serão
  % utilizadas.
  \item[FIXME] Indica que algo deve ser consertado.
  \item[TODO] Indica que algo deve ser feito.
\end{description}

%
% FIXME Comentar a linha abaixo de não for apresentar os
% símbolos utilizados.
\chapter*{S\'{i}mbolos}
\addcontentsline{toc}{chapter}{S\'{i}mbolos}

%%TODO Escrever o exemplo de símbolos.

%
% FIXME Comentar a linha abaixo de não for apresentar o
% glossário.
\chapter{Gloss\'{a}rio}

\begin{description}
    % FIXME Remover as duas explica\c{c}\~oes abaixo e incluir as que ser\~{a}o
    % utilizadas.
    \item[Tese] Proposi\c{c}\~ao que \'e enunciada e sustentada.
    \item[Disserta\c{c}\~ao] Discurso, exposi\c{c}\~ao ou exame minucioso de
        determinado assunto.
\end{description}

%
% As paginas com o conteúdo da tese são numeradas com algoritmos arábicos.
\mainmatter
%
% FIXME Remover as 3 linhas abaixo.
\input{ex_cap1}
\chapter{Título 2}
Esse é o segundo capítulo da sua defesa.

\section{Figuras}
É possível inserir figuras na sua defesa utilizando o comando
\lstinline+\includegraphics{caminho_figura}+. \\
\includegraphics{figuras/unicamp-logo.jpg}

É recomendado deixar as figuras\index{figura} como objetos flutuantes, para isso
utiliza-se o ambiente \lstinline+figure+.
\begin{figure}[!htb]
  \center
  \includegraphics[width=1cm]{figuras/unicamp-logo.jpg}
  \caption{Logo da UNICAMP.}
  \label{fig:log_unicamp}
\end{figure}

Além de inserir figuras é possível ``desenhar''\index{figura!desenhar} figuras
utilizando o pacote TikZ\index{figura!TikZ}, como na
Figura~\ref{fig:exem_tikz1}~e~\ref{fig:exem_tikz2}.
\begin{figure}[!htb]
  \centering
  \begin{tikzpicture}
    \draw (0,0) -- (0,2) -- (2,2) -- (2,0) -- (0,0);
    \fill (5,0) circle (1);
  \end{tikzpicture}
  \caption{Exemplo do uso do TikZ.}
  \label{fig:exem_tikz1}
\end{figure}
\begin{figure}[!htb]
  \centering
  \begin{tikzpicture}
    \node[draw] (A) at (0,1) {$A$};
    \node[draw,circle] (B) at (0,-1) {$B$};
    \node (C) at (1,0) {$C$};
    \node[fill=red,text=black] (D) at (3,0) {$D$};
    \node[draw,rectangle,fit=(A) (B) (C)] {};

    \draw[->] (A) -- (D);
    \draw[->] (B) -- (D);
    \draw[->] (C) -- (D);

    \coordinate (a) at (5,1);
    \coordinate (b) at (5,-1);

    \draw[dashed] (D) -- (a);
    \draw[dotted] (D) -- (b);
    \draw[<->] (a) -- (b);
  \end{tikzpicture}
  \caption{Outro exemplo do uso do TikZ.}
  \label{fig:exem_tikz2}
\end{figure}

\section{Tabelas}
Além de figuras também é possível inserir tabelas utilizando o ambiente
\lstinline+tabular+. \\
\begin{tabular}{cc}
  \toprule
  Tabela & Tabela \\
  \midrule
  Tabela & Tabela \\
  \bottomrule
\end{tabular}

Assim como as figuras, é recomendado que as tabelas sejam incluídas como
objetos flutuantes, para isso utiliza-se o ambiente \lstinline+table+.
\begin{table}[!htb]
  \caption{Exemplo de Tabela.}
  \label{tab:exem}
  \centering
  \begin{tabular}{cc}
    \toprule
    Tabela & Tabela \\
    \midrule
    Tabela & Tabela \\
    \bottomrule
  \end{tabular}
\end{table}

Além de inserir tabelas manualmente, como na Tabela~\ref{tab:exem}, também é
possível utilizar arquivos \emph{.csv} para criar tabelas. Neste caso, é
necessário incluir o pacote \texttt{csvsimple} que não foi incluído por questão
de compatibilidade com distribuições do \LaTeX \ anteriores a 2009.

\chapter{Título 3}
Esse é o terceiro capítulo da sua tese.

\section{Algoritmos}
Para facilitar a leitura, é bom incluir o algoritmo\index{algoritmo} de forma
adequada e não como uma lista, para isso utilize o ambiente
\lstinline+algorithmic+.
\begin{algorithmic}[2]
  \STATE $i \leftarrow 0$
  \STATE $y \leftarrow 0$
  \FOR{$i \leq 10$}
    \STATE $y \leftarrow y + i$
    \STATE $i \leftarrow i + 1$
  \ENDFOR
\end{algorithmic}

É possível nomear os algoritmos\index{algoritmo} ao utilizar o ambiente
\lstinline+algorithm+.
\begin{algorithm}
  \caption{Loop infinito.}
  \label{alg:loop_inf}
  \begin{algorithmic}
    \REQUIRE $n \geq 0$
    \ENSURE $y = 1 + 2 + \ldots + n$
    \STATE $y \leftarrow 0$
    \STATE $i \leftarrow 0$
    \IF{$n < 0$}
      \PRINT Entrada inadequada.
    \ELSE
      \WHILE{$i \neq n$}
        \STATE $y \leftarrow y + i$
      \ENDWHILE
    \ENDIF
  \end{algorithmic}
\end{algorithm}

Recomenda-se traduzir as instruções utilizadas no algoritmo. Para isso, é
preciso dar uma olhada na documentação do pacote \lstinline+algorithmic+.

\section{Códigos}
Em alguns casos, recomenda-se incluir um trecho de
código\index{codigo@código} e para isso utiliza-se o comando
\lstinline+\lstinputlisting{codigo}+.
\lstinputlisting[firstline=5, lastline=5, nolol=true]{src/exem.c}

Assim como os algoritmos também é possível nomear os códigos.
\lstinputlisting[firstline=10, lastline=12, caption=Loop em C,
language=C,  frame=single]{src/exem.c}

\chapter{Título 4}
Esse é o terceiro capítulo da sua tese.

\section{Referência bibliográfica}
Para a referência bibliográfica é utilizado o BibTeX e o pacote
\lstinline+biblatex+. Poderia ser utilizado apenas o BibTeX mas existem algumas
funcionalidades disponibilizadas no pacote \lstinline+biblatex+ que são muito
convenientes.

\subsection{BibTeX}
BibTeX é o nome de um formato de ``banco de dados'' para referências
bibliográficas desenvolvido para ser utilizado em conjunto com o LaTeX e também
do programa responsável por processar esse ``banco de dados''.

O ``banco de dados'' corresponde a um arquivo de texto com a extensão
\lstinline+.bib+. Por padrão este modelo utiliza o arquivo \lstinline+tese.bib+
que já encontra-se com algumas entradas para servirem de exemplo.

Cada referência no BibTeX segue a seguinte estrutura:
\begin{lstlisting}
@TIPO_DOCUMENTO{identificador,
campo1 = {valor do campo 1},
campo2 = {valor do campo 2},
campo3 = {valor do campo 3},
...
}
\end{lstlisting}

Para saber quais sobre os \lstinline+TIPO_DOCUMENTO+ existentes e sobre os
campos recomenda-se a documentação do BibTeX que pode ser acessada pelo comando:
\begin{lstlisting}
$ texdoc bibtex
\end{lstlisting}

Uma das grandes vantagens de se utilizar o BibTeX é que as chances de encontrar
o BibTeX de algum material na internet é extremamente alta. Tanto o Google
Scholar como o Google Books disponibilizam o BibTeX para todos os materiais
indexados em suas respectivas bases de dados.

\subsection{\lstinline+biblatex+}
Para que uma entrada do \lstinline+tese.bib+ seja incluído na referência
bibliográfica ele precisa ser utilizado em algum dos arquivos \lstinline+.tex+
que compõe sua dissertação/tese. Para utilizar uma referência utiliza-se uma das
variantes do comando \lstinline+\cite{id}+, onde
\lstinline+id+ correspode ao \lstinline+identificador+ utilizado na entrada do
BibTeX para a referência desejada.

O comando \lstinline+\cite{id}+ insere o número da referência entre colchetes,
como mostrado abaixo:
\begin{table}[!h]
  \centering
  \begin{tabular}{lc}
    \toprule
    Comando & Resultado \\ \midrule
    \lstinline+\cite{Swa82}+ & \cite{Swa82} \\
    \lstinline+\cite{Bailey}+ & \cite{Bailey} \\
    \lstinline+\cite{Ta}+ & \cite{Ta} \\
    \lstinline+\cite{Hale}+ & \cite{Hale} \\ \bottomrule
  \end{tabular}
\end{table}

Para inserir o nome dos autores e o número da referência entre colchetes,
utiliza-se o comando \lstinline+\textcite{id}+, como mostrado abaixo:
\begin{table}[!h]
  \centering
  \begin{tabular}{ll}
    \toprule
    Comando & Resultado \\ \midrule
    \lstinline+\textcite{Swa82}+ & \textcite{Swa82} \\
    \lstinline+\textcite{Bailey}+ & \textcite{Bailey} \\
    \lstinline+\textcite{Ta}+ & \textcite{Ta} \\
    \lstinline+\textcite{Hale}+ & \textcite{Hale} \\ \bottomrule
  \end{tabular}
\end{table}

Para inserir apenas o nome dos autores utiliza-se o comando
\lstinline+\citeauthor{id}+, como mostrado abaixo:
\begin{table}[!h]
  \centering
  \begin{tabular}{ll}
    \toprule
    Comando & Resultado \\ \midrule
    \lstinline+\citeauthor{Swa82}+ & \citeauthor{Swa82} \\
    \lstinline+\citeauthor{Bailey}+ & \citeauthor{Bailey} \\
    \lstinline+\citeauthor{Ta}+ & \citeauthor{Ta} \\
    \lstinline+\citeauthor{Hale}+ & \citeauthor{Hale} \\ \bottomrule
  \end{tabular}
\end{table}

Para inserir apenas o título da referência utiliza-se o comando
\lstinline+\citetitle{id}+, como mostrado abaixo:
\begin{table}[!h]
  \centering
  \begin{tabular}{ll}
    \toprule
    Comando & Resultado \\ \midrule
    \lstinline+\citetitle{Swa82}+ & \citetitle{Swa82} \\
    \lstinline+\citetitle{Bailey}+ & \citetitle{Bailey} \\
    \lstinline+\citetitle{Ta}+ & \citetitle{Ta} \\
    \lstinline+\citetitle{Hale}+ & \citetitle{Hale} \\ \bottomrule
  \end{tabular}
\end{table}

Para inserir apenas o ano de publicação da referência utiliza-se o comando
\lstinline+\citeyear{id}+, como mostrado abaixo:
\begin{table}[!h]
  \centering
  \begin{tabular}{lc}
    \toprule
    Comando & Resultado \\ \midrule
    \lstinline+\citeyear{Swa82}+ & \citeyear{Swa82} \\
    \lstinline+\citeyear{Bailey}+ & \citeyear{Bailey} \\
    \lstinline+\citeyear{Ta}+ & \citeyear{Ta} \\
    \lstinline+\citeyear{Hale}+ & \citeyear{Hale} \\ \bottomrule
  \end{tabular}
\end{table}

Para citações múltiplas, utiliza-se os comandos \lstinline+\cites{id1,id2,id3}+
ou \lstinline+\textcites{id1,id2,id3}+, como mostrado abaixo:
\begin{table}[!h]
  \centering
  \begin{tabular}{lc}
    \toprule
    Comando & Resultado \\ \midrule
    \lstinline+\cites{Bailey,Swa82}+ & \cites{Bailey,Swa82} \\
    \lstinline+\cites{Hale,Ta}+ & \cites{Hale,Ta} \\
    \lstinline+\textcites{Bailey,Swa82}+ & \textcites{Bailey,Swa82} \\
    \lstinline+\textcites{Hale,Ta}+ & \textcites{Hale,Ta} \\ \bottomrule
  \end{tabular}
\end{table}

Por último, caso deseje incluir uma referência na referência bibliográfica mas
suprimí-la ao longo do texto você deve utilizar o comando
\lstinline+\nocite{id}+.

% TODO Inserir os arquivos referentes ao corpo da tese.
%
% FIXME Se não for utilizar apêndices, comentar a linha abaixo.
\appendix
% FIXME Remover a linha abaixo.
\chapter{Título do Apêndice}
Esse é um apêndice da sua tese.

Como informado no ``Manual de Normalização para o NITEG e o PPGCI da ECI-UFMG''
disponível em
\url{http://ppgci.eci.ufmg.br/normalizacao/?Reda%E7%E3o_e_Estilo:Anexo_e_ap%EAndice}:
\begin{quote}
  Apêndices são os materiais elaborados pelo próprio autor com objetivo de
  completar uma argumentação. 
\end{quote}

\section{Seção A.1}
Essa é uma seção de um dos apêndices da sua tese.

\chapter{Git}
Git é o atual \textit{state-of-the-art} sistema de controle de versão e tem sido
utilizado em vários projetos, muitos na área de desenvolvimento de software,
sendo o grande destaque o Kernel Linux. Este modelo de tese possui suas versões
controladas por meio do Git e nas próximas sessões será apresentado algumas
dicas de como utilizar o Git para aproveitar melhor este modelo.

\section{Baixando o modelo}
Para baixar o modelo utilizando o Git você deve executar o comando:
\begin{lstlisting}[escapechar=@]
$ git clone https://github.com/r@-@gaia@-@cs/modelo_tese_imecc.git
\end{lstlisting}
Será criado uma pasta \lstinline+modelo_tese_imecc+ com os arquivos do
modelo.

Além da versão oficial do modelo existem algumas outras e para ver uma lista
destas versões você deve executrar o coamndo:
\begin{lstlisting}
$ git branch
\end{lstlisting}
Para selecionar a versões a ser utilizada, execute o comando:
\begin{lstlisting}
$ git branch versao_desejada
\end{lstlisting}
A versão oficial é chamada de \lstinline+master+.

Antes de começar a editar os arquivos, execute os comandos:
\begin{lstlisting}
$ git branch meu_trabalho
$ git checkout meu_trabalho
\end{lstlisting}

\section{Editando o modelo}
Evite ao máximo modificar os arquivos originais do modelo pois isso facilitará a
atualização do mesmo como será apresentado na seção seguinte.

Para adicionar novos arquivos ao controle de versão, execute o comando:
\begin{lstlisting}
$ git add novo_arquivo
\end{lstlisting}
E para adicionar modificações nos arquivos previamente adicionados ao controle
de versão, execute o comando:
\begin{lstlisting}[escapechar=@]
$ git add @-@u
\end{lstlisting}

Para criar uma nova versão da sua dissertação/tese, execute o comando:
\begin{lstlisting}[escapechar=@]
$ git commit @-@m 'Breve descricao do que foi feito ate agora.'
\end{lstlisting}

\section{Atualizando o modelo}
Como este modelo é um trabalho em progresso e não existe nenhuma garantia de que
as deliberações da Comissão Central de Pós-Graduação serão mantidas até que você
termine seu mestrado/doutorado é importante existir uma maneira de você
atualizar sua dissertação/tese já em processo de escrita com as novas
deliberações.

Antes de você atualizar o modelo, execute o seguinte comando:
\begin{lstlisting}[escapechar=@]
$ git commit @-@m 'Preparacao para atualizacao do modelo.'
\end{lstlisting}
Para atualizar o modelo, execute os comandos:
\begin{lstlisting}
$ git fetch
$ git checkout versao_desejada
$ git merge origin/versao_desejada
$ git checkout meu_trabalho
$ git rebase meu_trabalho versao_desejada
\end{lstlisting}

% TODO Inserir os arquivos referentes aos apêndices.
%
% TODO Se não for utilizar anexos, comentar as duas linhas abaixo.
\renewcommand{\appendixname}{Anexo}
\appendix
% FIXME Remover a linha abaixo.
\chapter{Título do Anexo}
Esse é um anexo da sua tese.

Como informado no ``Manual de Normalização para o NITEG e o PPGCI da ECI-UFMG''
disponível em
\url{http://ppgci.eci.ufmg.br/normalizacao/?Reda%E7%E3o_e_Estilo:Anexo_e_ap%EAndice}:
\begin{quote}
  Anexos são materiais não elaborados pelo autor, que servem de fundamentação,
  comprovação e ilustração, como mapa, leis, estatutos entre outros.
\end{quote}

% TODO Inserir os arquivos referentes aos apêndices e anexos.
%
\backmatter
\printbibliography
\addcontentsline{toc}{chapter}{Referência Bibliográfica}
\clearpage
\printindex
\addcontentsline{toc}{chapter}{Índice Remissivo}
\end{document}
