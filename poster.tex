\documentclass[]{beamer}
%%%%%%%%%%%%%%%%%%%%%%%%%%%% Configuração: pacotes %%%%%%%%%%%%%%%%%%%%%%%%%%%%
%%%%%%%%%%%%%%%%%%%%%%%%%%%%%% Pacotes: básicos %%%%%%%%%%%%%%%%%%%%%%%%%%%%%%
\usepackage[utf8]{inputenc}
\usepackage{cmap}
\usepackage[T1]{fontenc}
\usepackage[english,portuguese]{babel}
\usepackage{indentfirst}
\usepackage{beamerposter}
\usepackage{geometry}
\usepackage{biblatex}
\addbibresource{biblatex_style_samples/sample.bib}
\addbibresource{biblatex_style_samples/sample-abel.bib}


%%%%%%%%%%%%%%%%%%%%%%%%%%%%%%% Pacotes: links %%%%%%%%%%%%%%%%%%%%%%%%%%%%%%%
\usepackage{url}
\usepackage{breakurl}
\usepackage{hyperref}
% FIXME Comente o pacote abaixo quando for concluir sua defesa e for entregar a
% versão final.
\usepackage{showkeys}


%%%%%%%%%%%%%%%%%%%%%%%%%%%%%%%% Pacotes: ams %%%%%%%%%%%%%%%%%%%%%%%%%%%%%%%%
\usepackage{amsmath}
\usepackage{amsfonts}
\usepackage{amssymb}
\usepackage{amsthm}
\usepackage{breqn}


%%%%%%%%%%%%%%%%%%%%%%%%%%%%%% Pacotes: tabelas %%%%%%%%%%%%%%%%%%%%%%%%%%%%%%
\usepackage{multicol}
\usepackage{multirow}
\usepackage{array}
\usepackage{booktabs}


%%%%%%%%%%%%%%%%%%%%%%%%%%%%%% Pacotes: cores %%%%%%%%%%%%%%%%%%%%%%%%%%%%%%%%
%\usepackage[usenames,dvipsnames,svgnames,table]{xcolor}


%%%%%%%%%%%%%%%%%%%%%%%%%%%%%% Pacotes: figuras %%%%%%%%%%%%%%%%%%%%%%%%%%%%%%
\usepackage{pdfpages}
\usepackage{graphicx}
\usepackage{tikz}
\usetikzlibrary{fit}
\usepackage{wrapfig}


%%%%%%%%%%%%%%%%%%%%%%%%%%%%% Pacotes: algoritmos %%%%%%%%%%%%%%%%%%%%%%%%%%%%%
\usepackage{algorithmic}
\usepackage{algorithm}
\floatname{algorithm}{Algoritmo}
\renewcommand{\listalgorithmname}{Lista de Algoritmos}


%%%%%%%%%%%%%%%%%%%%%%%%%%%%%% Pacotes: códigos %%%%%%%%%%%%%%%%%%%%%%%%%%%%%%
\usepackage{textcomp}
\usepackage{listings}
\renewcommand\lstlistingname{Código}
\renewcommand\lstlistlistingname{Lista de Códigos}


%%%%%%%%%%%%%%%%%%%%%%%%%%%%%%% Pacotes: index %%%%%%%%%%%%%%%%%%%%%%%%%%%%%%%
\usepackage{makeidx}
\makeindex


%%%%%%%%%%%%%%%%%%%%%%%%%%%%%%% Pacotes: fontes %%%%%%%%%%%%%%%%%%%%%%%%%%%%%%
\usepackage{lmodern}
\usepackage{mathrsfs}


% TODO Inserir pacotes adicionais aqui.
  % Arquivo com os pacotes.
\geometry{paperwidth=90cm,paperheight=100cm}

%%%%%%%%%%%%%%%%%%%%%%%%%%% Configuração: definições %%%%%%%%%%%%%%%%%%%%%%%%%%%
%%%%%%%%%%%%%%%%%%%%%%%%%%%%% Configurações: links %%%%%%%%%%%%%%%%%%%%%%%%%%%%%
\hypersetup{
hypertexnames=false,
}


%%%%%%%%%%%%%%%%%%%%%%%%%% Configurações: numeração %%%%%%%%%%%%%%%%%%%%%%%%%%


%%%%%%%%%%%%%%%%%%%%%%%%%%%%% Configurações: ams %%%%%%%%%%%%%%%%%%%%%%%%%%%%%
\theoremstyle{definition}
\newtheorem{thm}{Teorema}[section]
\newtheorem{con}[thm]{Conjectura}
\newtheorem{cor}[thm]{Corolário}
\newtheorem{dfn}[thm]{Definição}
\newtheorem{exm}[thm]{Exemplo}
\newtheorem{lem}[thm]{Lema}
\newtheorem{obs}[thm]{Observação}
\newtheorem{pps}[thm]{Proposição}


%%%%%%%%%%%%%%%%%%%%%%%%%%%%%% Configurações: códigos %%%%%%%%%%%%%%%%%%%%%%%%
\lstset{
basicstyle=\ttfamily,
keywordstyle=\bfseries,
breaklines=false
}


% TODO Inserir configurações adicionais aqui.
  % Arquivo com algumas configurações.

%%%%%%%%%%%%%%%%%%%%%%%%%%%%%%% Início do poster %%%%%%%%%%%%%%%%%%%%%%%%%%%%%%%
\begin{document}
\begin{frame}[t,fragile]
%%%%%%%%%%%%%%%%%%%%%%%%%%%%%%% Título do poster %%%%%%%%%%%%%%%%%%%%%%%%%%%%%%%
% Com base em http://www.prp.unicamp.br/pibic/congresso_formatacaopaineis.php
  \begin{center}
    \begin{huge}
      % FIXME Substituir 'Título da defesa' pelo título da defesa.
      Título da defesa

      \vspace{20pt}
      Instituto de Matemática, Estatística e Computação Científica
    \end{huge}

    \vspace{20pt}
    \begin{Large}
      \begin{tabular}[]{c}
        % FIXME Substituir 'Nome completo do aluno' pelo seu nome.
        Nome completo do aluno \\
        % FIXME Substituir 'email_aluno@ime.unicamp.br' pelo seu email.
        \url{email_aluno@ime.unicamp.br}
      \end{tabular} \hspace{5cm}
      \begin{tabular}[]{c}
        % FIXME Substituir 'Nome completo do orientador' pelo nome completo do seu
        % orientador.
        Nome completo do orientador \\
        % FIXME Substituir 'email_orientador@ime.unicamp.br' pelo email do seu
        % orientador.
        \url{email_orientador@ime.unicamp.br}
      \end{tabular}
      % FIXME Remover as 9 linhas abaixo se não tiver um coorientador.
      \hspace{5cm}
      \begin{tabular}[]{c}
        % FIXME Substituir 'Nome completo do coorientador' pelo nome completo do seu
        % coorientador.
        Nome completo do coorientador \\
        % FIXME Substituir 'email_coorientador@ime.unicamp.br' pelo email do seu
        % coorientador.
        \url{email_coorientador@ime.unicamp.br}
      \end{tabular}
    \end{Large}
  \end{center}
  \vspace{20pt}

%%%%%%%%%%%%%%%%%%%%%%%%%%%%%%% Corpo do poster %%%%%%%%%%%%%%%%%%%%%%%%%%%%%%%
% O ambiente columns é definido na classe beamer.
  \begin{columns}[t]
    \begin{column}{0.4\textwidth}
      O pacote \lstinline+beamerposter+ altera o tamanho das fontes de acordo com
      a classe \lstinline+a0poster+:
      \begin{center}
        \begin{tabular}[]{|l|r|r|}
          \hline
          Fonte & Tamanho (pt) & Tamanho (mm) \\ \hline
          \lstinline+\tiny+         & 12pt    & 4.2mm \\ \hline
          \lstinline+\scriptsize+   & 14.4pt  & 5.0mm \\ \hline
          \lstinline+\footnotesize+ & 17.28pt & 6.1mm \\ \hline
          \lstinline+\small+        & 20.74pt & 7.3mm \\ \hline
          \lstinline+\normalsize+   & 24.88pt & 8.7mm \\ \hline
          \lstinline+\large+        & 29.86pt & 10.5mm \\ \hline
          \lstinline+\Large+        & 35.83pt & 11.9mm \\ \hline
          \lstinline+\LARGE+        & 43pt    & 15.1mm \\ \hline
          \lstinline+\huge+         & 51.6pt  & 18.2mm \\ \hline
          \lstinline+\Huge+         & 61.92pt & 21.8mm \\ \hline
          \lstinline+\veryHuge+     & 74.3pt  & 26.2mm \\ \hline
          \lstinline+\VeryHuge+     & 89.16pt & 31.4mm \\ \hline
          \lstinline+\VERYHuge+     & 107pt   & 37.7mm \\ \hline
        \end{tabular}
      \end{center}
      Abaixo você encontra uma ilustração do tamanho das fontes acima descritas:
      \begin{center}
        {\tiny tiny}\par
        {\scriptsize scriptsize}\par
        {\footnotesize footnotesize}\par
        {\normalsize normalsize}\par
        {\large large}\par
        {\Large Large}\par
        {\LARGE LARGE}\par
        {\veryHuge
        veryHuge}\par
        {\VeryHuge
        VeryHuge}\par
        {\VERYHuge
        VERYHuge}\par
      \end{center}

      Para equações matemáticas está sendo utilizado os seguintes pacotes:
      \begin{itemize}
        \item amsmath,
        \item amsfonts,
        \item amssymb,
        \item amsthm,
        \item breqn.
      \end{itemize}

      Um exemplo de equação na mesma linha:
      $ 1 + 1 + 1 + 1 + 1 + 1 = 6$,
      que é trivial\index{trivial} de ser verificada.

      Equações na mesma linha são quebradas automaticamente:
      $ 1 + 1 + 1 + 1 + 1 + 1 + 1 + 1 + 1 + 1 + 1 + 1 = 12$.

      Além de equações na mesma linha, também é possível utilizar equações em
      destaque:
      \begin{dmath}
        1 + 1 + 1 + 1 + 1 + 1 + 1 + 1 + 1 + 1 + 1 + 1 + 1 + 1 + 1 + 1 + 1 + 1 + 1
        + 1 + 1 + 1 + 1 + 1 + 1 + 1 + 1 + 1 + 1 + 1 = 30
      \end{dmath}

      No caso do desenvolvimento/simplificação de uma expressão
      matemática também é recomendado utilizar os ambientes
      disponibilizados pelo pacote breqn.
      \begin{dmath}
        f(x) = 1 + 1 + 1 + 1 + 1 + 1 + 1 + 1 + 1 + 1 + 1 + 1 + 1 + 1 + 1 + 1 + 1 +
        1 + 1 + 1 + 1 + 1 + 1 + 1 + 1 + 1 + 1 + 1 + 1 + 1
        = 2 + 2 + 2 + 2 + 2 + 2 + 2 + 2 + 2 + 2 + 2 + 2 + 2 + 2 + 2
        = 4 + 4 + 4 + 4 + 4 + 4 + 4 + 2
        = 8 + 8 + 8 + 6
        = 16 + 14
        = 30
      \end{dmath}

      Uma das vantagens de utilizar a classe \lstinline+beamer+ para o poster é
      que caixas de texto, como a ilustrada abaixo, já estão definidas.

      \begin{block}{Exemplo de caixa de texto}
        Exemplo de texto dentro de uma caixa. Essa é uma das vantagens de
        utilizar a classe \lstinline+beamer+.
      \end{block}

      As caixas de texto também são utilizadas para alguns ambientes
      matemáticos.

      \begin{thm}
        Teorema, Teorema, Teorema, Teorema.
      \end{thm}

      \begin{con}
        Conjectura, Conjecture, Conjectura, Conjectura.
      \end{con}

      \begin{cor}
        Corolário, Corolário, Corolário, Corolário.
      \end{cor}

      \begin{dfn}
        Definição, Definição, Definição.
      \end{dfn}

      É possível inserir figuras. Recomenda-o não fazer como objetos flutuantes
      para evitar problemas.
      \begin{center}
        \includegraphics[width=0.1\textwidth]{figuras/unicamp-logo}
      \end{center}
    \end{column}
    \begin{column}{0.4\textwidth}
      Além de inserir figuras é possível ``desenhar'' figuras
      utilizando o pacote TikZ\index{figura!TikZ}.
      \begin{center}
        \begin{tikzpicture}
          \node[draw] (A) at (0,1) {$A$};
          \node[draw,circle] (B) at (0,-1) {$B$};
          \node (C) at (1,0) {$C$};
          \node[fill=red,text=black] (D) at (3,0) {$D$};
          \node[draw,rectangle,fit=(A) (B) (C)] {};

          \draw[->] (A) -- (D);
          \draw[->] (B) -- (D);
          \draw[->] (C) -- (D);

          \coordinate (a) at (5,1);
          \coordinate (b) at (5,-1);

          \draw[dashed] (D) -- (a);
          \draw[dotted] (D) -- (b);
          \draw[<->] (a) -- (b);
        \end{tikzpicture}
      \end{center}

      Assim como as figuras, recomenda-se não inserir tabelas como objetos
      flutuantes.

      Também é possível incluir algoritmos que para facilitar a leitura é bom
      ser incluidor de forma adequada e não como uma lista. Para isso utilize o
      ambiente \lstinline+algorithmic+.
      \begin{algorithmic}[2]
        \STATE $i \leftarrow 0$
        \STATE $y \leftarrow 0$
        \FOR{$i \leq 10$}
        \STATE $y \leftarrow y + i$
        \STATE $i \leftarrow i + 1$
        \ENDFOR
      \end{algorithmic}

      Além de algoritmos também é possível inserir trechos de códigos de
      computador. Ao fazer isso recomenda-se utilizar o comando
      \lstinline+\lstinputlisting{codigo}+.
      \lstinputlisting[firstline=5, lastline=5, nolol=true]{src/exem.c}

      Se possível, também é interessante colocar algumas poucas referências
      bibliográficas, apenas as mais relevantes. Para citar uma das referências
      utiliza-se o comando \lstinline+\cite{label}+, como por exemplo:
      \cite{Bailey},~\cite{Bay1}.

      \textbf{Referências bibliográficas:}

      \printbibliography

      \textbf{Adicionando cores}

      Este modelo não utiliza nenhum esquema de cores ou tema. Para selecionar
      um esquema de cores utiliza-se o comando \lstinline+\usecolortheme{tema}+
      que deve ser adicionado no final do arquivo
      \lstinline+poster-configuracao.tex+. Alguns temas de cores são:
      \begin{lstlisting}
\usecolortheme{albatross}
\usecolortheme{crane}
\usecolortheme{wolverine}
\usecolortheme{orchid}
      \end{lstlisting}
      E para selecionar o tema utiliza-se o comando
      \lstinline+\usetheme{tema}+ que também deve ser adicionado no final do
      arquivo \lstinline+poster-configuracao.tex+. Alguns temas são:
      \begin{lstlisting}
\usetheme{Bergen}
\usetheme{Madrid}
\usetheme{AnnArbor}
\usetheme{CambridgeUS}
      \end{lstlisting}
      Ao escolher um tema deve-se tomar cuidado pois muitos deles possuem
      configurações para cabeçalho e rodapé que são indesejados em um poster.

      Se desejar personalizar a sua maneira o poster, recomenda-se dar uma lida no
      manual da classe \lstinline+beamer+.
    \end{column}
  \end{columns}
  \vfill
  % FIXME Remova as 7 linhas a seguir se não desejar licenciar o poster sob
  % CC-BY.
  \begin{center}
    \begin{tabular}[]{cc}
      \multirow{2}{*}{\includegraphics[height=60pt]{figuras/cc-by}} &
      \Large{Exceto indicado o contrário, este trabalho é licenciado sob}\\
      & \Large{\url{http://creativecommons.org/licenses/by/3.0/}.}
    \end{tabular}
  \end{center}
  % FIXME Descomente as 7 linhas a seguir se não desejar licenciar o poster sob
  % CC-BY-SA.
% \begin{center}
%   \begin{tabular}[]{cc}
%     \multirow{2}{*}{\includegraphics[height=60pt]{figuras/cc-by-sa}} &
%     \Large{Exceto indicado o contrário, este trabalho é licenciado sob}\\
%     & \Large{\url{http://creativecommons.org/licenses/by-sa/3.0/}.}
%   \end{tabular}
% \end{center}
\end{frame}
\end{document}
\end{frame}
\end{document}
