% FIXME Ao final, deixe descomentada a linha correspondente ao numero de paginas
% que a sua defesa possue.
\documentclass[12pt, letterpaper, oneside]{book}  % Para menos de 100 paginas.
% \documentclass[12pt, letterpaper, twoside]{book}  % Para mais de 100 paginas.

%%%%%%%%%%%%%%%%%%%%%%%%%%%% Configuracao: pessoais %%%%%%%%%%%%%%%%%%%%%%%%%%%% 
% FIXME Substituir 'Nome completo do aluno' pelo seu nome.
\newcommand{\autor}{Nome completo do aluno}
% FIXME Se for do sexo feminino, decomente a linha a seguir.
% \def\femaleAuthor{}

% FIXME Substituir 'Titulo da defesa' pelo t�tulo da defesa.
\newcommand{\titulo}{T\'{i}tulo da defesa}
% FIXME Se estiver no programa de mestrado, decomente a linha a seguir.
% \def\mestrado{}
% FIXME Deixe descomente apenas a linha referente ao departamento.
% \def\matematica{}
\def\aplicada{}
% \def\estatistica{}

% FIXME Substituir 'Nome completo do orientador' pelo nome completo do seu
% orientador.
\newcommand{\orientador}{Nome completo do orientador}
% FIXME Se for orientado por uma mulher, decomente a linha a seguir.
% \def\femaleOrientador{}

% FIXME Substituir 'Nome completo do coorientador' pelo nome completo do seu
% coorientador. Caso n�o tenha coorientador, comente a linha a seguir.
\newcommand{\coorientador}{Nome completo do coorientador}
% FIXME Se for coorientado por uma mulher, decomente a linha a seguir.
% \def\femaleCoorientador{}

% FIXME Substituir 'Ano' pelo ano em que ocorreu sua defesa.
\newcommand{\ano}{Ano}

%%%%%%%%%%%%%%%%%%%%%%%%% Configuracao: comportamento %%%%%%%%%%%%%%%%%%%%%%%%%
%%%%%%%%%%%%%%%%%%%%%%%%%%%%%% Pacotes: básicos %%%%%%%%%%%%%%%%%%%%%%%%%%%%%% 
% FIXME Selecione a codificação adequada.
\usepackage[utf8]{inputenc}
% \usepackage[latin1]{inputenc}
\usepackage[T1]{fontenc}
%%TODO Usar fullpage ou geometry?
\usepackage[cm]{fullpage}
%\usepackage[top=3cm, bottom=3cm, left=2cm, right=2cm]{geometry}
\usepackage[brazilian]{babel}
%%TODO Usar espaço na primeira linha?
\usepackage{indentfirst}
%%%%%%%%%%%%%%%%%%%%%%%%%%%%%%% Pacotes: links %%%%%%%%%%%%%%%%%%%%%%%%%%%%%%%
\usepackage[hyphens]{url}
\usepackage{hyperref}
\usepackage{breakurl}
%%%%%%%%%%%%%%%%%%%%%%%%%%%%%%%% Pacotes: ams %%%%%%%%%%%%%%%%%%%%%%%%%%%%%%%% 
\usepackage{amsmath}
\usepackage{amsfonts}
\usepackage{amssymb}
\usepackage{amsthm}
%%%%%%%%%%%%%%%%%%%%%%%%%%%%%% Pacotes: tabelas %%%%%%%%%%%%%%%%%%%%%%%%%%%%%%
\usepackage{multicol}
\usepackage{multirow}
\usepackage{array}
\usepackage{csvsimple}
%%%%%%%%%%%%%%%%%%%%%%%%%%%%%% Pacotes: figuras %%%%%%%%%%%%%%%%%%%%%%%%%%%%%% 
\usepackage{pdfpages}
\usepackage{graphicx}
\usepackage{tikz}
\usepackage{wrapfig}
%%%%%%%%%%%%%%%%%%%%%%%%%%%%% Pacotes: algoritmos %%%%%%%%%%%%%%%%%%%%%%%%%%%%% 
\usepackage{algorithmic}
\usepackage[chapter]{algorithm}
\floatname{algorithm}{Algoritmo}
\renewcommand{\listalgorithmname}{Lista de Algoritmos}
%%%%%%%%%%%%%%%%%%%%%%%%%%%%%% Pacotes: códigos %%%%%%%%%%%%%%%%%%%%%%%%%%%%%% 
\usepackage{textcomp}
\usepackage{listings}
\renewcommand\lstlistingname{C\'{o}digo}
\renewcommand\lstlistlistingname{Lista de C\'{o}digos}
%%%%%%%%%%%%%%%%%%%%%%%%%%%%%%% Pacotes: index %%%%%%%%%%%%%%%%%%%%%%%%%%%%%%% 
\usepackage{makeidx}

%%%%%%%%%%%%%%%%%%%%%%%%%% Configurações: numeração %%%%%%%%%%%%%%%%%%%%%%%%%% 
\numberwithin{equation}{section}
\numberwithin{section}{chapter}
%%%%%%%%%%%%%%%%%%%%%%%%%%%%% Configurações: ams %%%%%%%%%%%%%%%%%%%%%%%%%%%%% 
\theoremstyle{definition}
\newtheorem{thm}{Teorema}[section]
\newtheorem{con}[thm]{Conjectura}
\newtheorem{cor}[thm]{Corol\'ario}
\newtheorem{dfn}[thm]{Defini\c c\~ao}
\newtheorem{exm}[thm]{Exemplo}
\newtheorem{lem}[thm]{Lema}
\newtheorem{obs}[thm]{Observa\c c\~ao}
\newtheorem{pps}[thm]{Proposi\c c\~ao}
%%%%%%%%%%%%%%%%%%%%%%%%%%%%% Configurações: index %%%%%%%%%%%%%%%%%%%%%%%%%%%%% 
\makeindex  % Necessário para gerar o index.
% TODO Inserir configurações adicionais aqui.
  % Arquivo com os pacotes.
%%%%%%%%%%%%%%%%%%%%%%%%%% Configurações: numeração %%%%%%%%%%%%%%%%%%%%%%%%%% 
\numberwithin{equation}{section}
\numberwithin{section}{chapter}


%%%%%%%%%%%%%%%%%%%%%%%%%%%%% Configurações: ams %%%%%%%%%%%%%%%%%%%%%%%%%%%%% 
\theoremstyle{definition}
\newtheorem{thm}{Teorema}[section]
\newtheorem{con}[thm]{Conjectura}
\newtheorem{cor}[thm]{Corol\'ario}
\newtheorem{dfn}[thm]{Defini\c c\~ao}
\newtheorem{exm}[thm]{Exemplo}
\newtheorem{lem}[thm]{Lema}
\newtheorem{obs}[thm]{Observa\c c\~ao}
\newtheorem{pps}[thm]{Proposi\c c\~ao}


% TODO Inserir configurações adicionais aqui.
  % Arquivo com algumas configuracoes.

%%%%%%%%%%%%%%%%%%%%%%%%%% Inicio do texto da defesa %%%%%%%%%%%%%%%%%%%%%%%%%%
\begin{document}
% WARNING Todas as paginas deverao ser numeradas.
%
% As paginas iniciais sao numeradas com algorimos romanos em sua forma
% minuscula.
\frontmatter
%
\vspace*{2cm}
\begin{center}
    % O tamanho da fonte deve ser 16pt.
    % Deve-se utilizar caixa alta.
    {\Large \scshape \autor}
\end{center}
\vspace{3cm}
\begin{center}
    % O tamanho da fonte deve ser 16pt em negrito.
    % Deve-se utilizar caixa alta.
    {\Large \scshape \bfseries \titulo}
\end{center}
\vspace{8cm}
\begin{center}
    % O tamanho da fonte deve ser 12pt em negrito.
    % Deve-se utilizar caixa alta.
    {\scshape \bfseries CAMPINAS \\ \ano}
\end{center}
  % Nao edite esse arquivo.
\newpage\mbox{}\thispagestyle{plain}\newpage  % Pagina em branco.
%
% WARNING A folha de rosto precisa ser assinada pelos orientadores.
% FIXME Substitua arquivo folha-de-rosto.pdf por uma copia escaneada, comente
% esta linha e descomente a proxima.
\thispagestyle{plain}
\pagenumbering{roman}
\setcounter{page}{1}

\begin{center}
    {\Large {\sc Universidade Estadual de Campinas} }
\end{center}

\begin{center}
    {\Large {\sc Instituto de Matem\'atica, Estat\'istica e Computa\c c\~ao
    Cient\'ifica} }
\end{center}

\vspace{2cm}

\begin{center}
    {\huge {\sc \autor} }
\end{center}

\vspace{2cm}

\begin{center}
    {\huge {\sc \titulo} }
\end{center}

\vspace{1cm}

\begin{flushright}
    \begin{minipage}[c]{7.5cm}
        {\large
        % FIXME Adeque dependendo do nível.
        % Disserta\c c\~ao de mestrado % descomente de for mestrado.
        Tese de doutorado % descomente de for doutorado.
        apresentada ao Instituto de Matem\'atica, Estat\'istica e
        Computa\c c\~ao Cient\'ifica da Universidade Estadual de Campinas para
        obten\c c\~ao do t\'itulo de
        % FIXME Adeque dependendo do nível.
        % mestre/a % descomente de for mestrado.
        doutor % descomente de for doutorado.
        em
        % FIXME Adeque dependendo da área.
        % Matem\'atica.
        Matem\'atica Aplicada.
        % Estat\'istica.
        }
    \end{minipage}
\end{flushright}

\vspace{1cm}

\begin{center}
    {\large {\sc
    Orientador: Prof. Dr. \orientador

    Co-orientador: Prof. Dr. \coorientador
    } }
\end{center}

\vspace{1cm}

\begin{center}
    \begin{minipage}[c]{0.8\textwidth}
        {\large
        Este exemplar corresponde \`a vers\~ao final da 
        % FIXME Adeque dependendo do nível.
        % tese % descomente se for doutorado.
        disserta\c c\~ao % descomente se for mestrado.
        defendida pelo aluno \autor e orientada pelo Prof. Dr. \orientador
        }
    \end{minipage}
\end{center}

\vspace{1cm}

\begin{center}
    \begin{minipage}[c]{8.5cm}
        \begin{center}
            {\sc
            \rule[1pt]{8.5cm}{.5pt} % Assinatura do orientador

            Prof. Dr. \orientador
            }
        \end{center}
    \end{minipage}
    \hspace{.5cm}
    \begin{minipage}[c]{8.5cm}
        \begin{center}
            {\sc
            \rule[1pt]{8.5cm}{.5pt} % Assinatura do co-orientador

            Prof. Dr. \coorientador
            }
        \end{center}
    \end{minipage}
\end{center}

\vspace{1cm}

\begin{center}
    {\Large {\sc Campinas \\ \ano} }
\end{center}

% \includepdf{folha-de-rosto.pdf}
%
% WARNING A ficha catalografica deve estar no verso da folha de rosto.
% FIXME O arquivo ficha-catalografica.pdf deve ser sobrescrito com uma c�pia
% do arquivo pdf que a bibliotaca lhe enviar.
\includepdf{ficha-catalografica}
%
% WARNING A folha de aprovacao deve ser assinada pelos membros da banca apos a
% defesa.
% FIXME Substitua o arquivo folha-de-aprovacao.pdf por uma copia escaneada.
\includepdf{folha-de-aprovacao}
%
% FIXME Se n�o for incluir a dedicat�ria, comentar a linha abaixo.
\chapter*{}
\begin{center}
    \emph{
    % FIXME Remover as duas linhas abaixo.
    Aos meus \ldots. \\
    Para \ldots.} 
    % TODO Inserir a dedicatória aqui.
\end{center}

%
% FIXME Se n�o for incluir a epigrafe, comentar a linha abaixo.
%%TODO Escrever o modelo da epigrafe.
\chapter*{}
\vfill
\begin{flushright}
  Epígrafe \\
  Epígrafe \\
  Epígrafe \\
  Epígrafe
\end{flushright}

%
% FIXME Se n�o for incluir os agradecimentos, comentar a linha abaixo.
\chapter*{Agradecimentos}
% FIXME Escrever os agradecimentos.
Aqui deve-se agradecer quem merece agradecimento

Se recebeu bolsa de algum \'org\~ao de fomento, n\~ao esque\c{c}a de agradec\^{e}-lo.

%
\chapter{Resumo}
\addcontentsline{toc}{chapter}{Resumo}
% FIXME Remover as 3 linhas abaixo.
resumo resumo resumo
resumo resumo resumo
resumo resumo resumo
% WARNING O resumo deve conter no máximo 500 palavras.
% TODO Inserir o resumo em portugês aqui.

\vspace{.5cm}
\textbf{Palavras-chave}:
% FIXME Remover a linha abaixo.
palavra01, palavra02, palavra03.
% TODO Inserir as palavras-chave aqui.

\chapter{Abstract}
% FIXME Remover as 3 linhas abaixo.
abstract abstract abstract
abstract abstract abstract
abstract abstract abstract
% WARNING O abstract deve conter no máximo 500 palavras.
% TODO Inserir o resumo em inglês aqui.

\vspace{.5cm}
\textbf{Keywords}:
% FIXME Remover a linha abaixo.
keyword01, keyword02, keyword03.
% TODO Inserir as palavras-chave em inglês aqui.

%
\tableofcontents
% FIXME Comentar as linhas abaixo de n�o desejar listar as figuras
% apresentadas.
\listoffigures
\addcontentsline{toc}{chapter}{Lista de Figuras}
%
% FIXME Comentar as linhas abaixo de n�o desejar listar as tabelas
% apresentadas.
\listoftables
\addcontentsline{toc}{chapter}{Lista de Tabelas}
%
% FIXME Comentar as linhas abaixo de n�o desejar listar os
% algoritmos apresentados.
\listofalgorithms
\addcontentsline{toc}{chapter}{Lista de Algoritmos}
%
% FIXME Comentar as linhas abaixo de n�o desejar listar os trechos de c�digo
% apresentados.
\lstlistoflistings
\addcontentsline{toc}{chapter}{Lista de C\'{o}digos}
%
% FIXME Comentar a linha abaixo de n�o for apresentar as
% abrevia��es utilizadas.
\chapter*{Lista de Abreviaturas e Siglas\markboth{Lista de Abreviaturas e Siglas}{}}  % \markboth{}{} é utilizado para corrigir o cabeçalho.

\begin{description}
  % FIXME Remover as duas abreviações/siglas abaixo e incluir as que serão
  % utilizadas.
  \item[FIXME] Indica que algo deve ser consertado.
  \item[TODO] Indica que algo deve ser feito.
\end{description}

%
% FIXME Comentar a linha abaixo de n�o for apresentar os
% s�mbolos utilizados.
\chapter*{S\'{i}mbolos}
\addcontentsline{toc}{chapter}{S\'{i}mbolos}

%%TODO Escrever o exemplo de símbolos.

%
% FIXME Comentar a linha abaixo de n�o for apresentar o
% glossario.
\chapter{Gloss\'{a}rio}

\begin{description}
    % FIXME Remover as duas explica\c{c}\~oes abaixo e incluir as que ser\~{a}o
    % utilizadas.
    \item[Tese] Proposi\c{c}\~ao que \'e enunciada e sustentada.
    \item[Disserta\c{c}\~ao] Discurso, exposi\c{c}\~ao ou exame minucioso de
        determinado assunto.
\end{description}

%
% As paginas com o conteudo da tese sao numeradas com algorimos arabicos.
\mainmatter
%
% FIXME Remover as 3 linhas abaixo.
\chapter{Título 1}
Esse é o primeiro capítulo da sua tese.

\section{Seção 1.1}
Essa é uma seção da sua tese.

\subsection{Subseção 1.1.1}

Essa é uma subseção da sua tese.

\subsection{Subseção 1.1.2}

Essa é outra subseção da sua tese.

\section{Equações matemáticas}
Para equações matemáticas está sendo utilizado os seguintes
pacotes:
\begin{itemize}
  \item amsmath,
  \item amsfonts,
  \item amssymb,
  \item amsthm,
  \item breqn.
\end{itemize}

Um exemplo de equação na mesma linha: 
$ 1 + 1 + 1 + 1 + 1 + 1 = 6$, 
que é trivial\index{trivial} de ser verificada.

Equações na mesma linha são quebradas automaticamente:
$ 1 + 1 + 1 + 1 + 1 + 1 + 1 + 1 + 1 + 1 + 1 + 1 = 12$. 

Ao utilizar equações na mesma linha deve-se tomar o cuidado de manter a
legibilidade da equação e não modificar a altura da linha. É
errado utilizar $\frac{1}{2} + \frac{1}{2} = 1$ ou $\frac{\partial}{\partial x}
(y^2 + 2xy + x^2) = 2y + 2x$, devendo ser utilizado $(1/2) + (1/2) = 1$ ou
$\partial (y^2 + 2 x y + x^2) / \partial x = 2 y + 2 x$.

Além de equações na mesma linha, também é possível
utilizar equações em destaque:
\begin{equation}
1 + 1 + 1 + 1 + 1 + 1 = 6
\end{equation}
ou
\begin{equation*}
1 + 1 + 1 + 1 + 1 + 1 = 6.
\end{equation*}
Ao utilizar equações matemáticas em destaque, não esqueça da pontuação nas
equações.

Recomenda-se utilizar os ambientes
\begin{itemize}
  \item dmath(*) e
  \item dgroup(*)
\end{itemize}
disponibilizados pelo pacote breqn ao invés dos ambientes
\begin{itemize}
  \item equation(*,)
  \item align(*),
  \item multiline(*) e
  \item split
\end{itemize}
pois os primeiros quebram e alinham automaticamente as equações em destaque.
Veja a seguir a ``equivalência'' entre os ambientes:
\begin{description}
  \item[dmath]
    \begin{dmath}
      1 + 1 + 1 + 1 + 1 + 1 + 1 + 1 + 1 + 1 + 1 + 1 + 1 + 1 + 1 + 1 + 1 + 1 + 1
      + 1 + 1 + 1 + 1 + 1 + 1 + 1 + 1 + 1 + 1 + 1 = 30
    \end{dmath}
  \item[equation] 
    \begin{equation}
      1 + 1 + 1 + 1 + 1 + 1 + 1 + 1 + 1 + 1 + 1 + 1 + 1 + 1 + 1 + 1 + 1 + 1 + 1
      + 1 + 1 + 1 + 1 + 1 + 1 + 1 + 1 + 1 + 1 + 1 = 30
    \end{equation}
  \item[equation com split] 
    \begin{equation}
      \begin{split}
        1 + 1 + 1 + 1 + 1 + 1 + 1 + 1 + 1 + 1 + 1 + 1 + 1 + 1 + 1 \\
        + 1 + 1 + 1 + 1 + 1 + 1 + 1 + 1 + 1 + 1 + 1 + 1 + 1 + 1 + 1 = 30 
      \end{split}
    \end{equation}
  \item[align] 
    \begin{align}
      1 + 1 + 1 + 1 + 1 + 1 + 1 + 1 + 1 + 1 + 1 + 1 + 1 + 1 + 1 \\
      + 1 + 1 + 1 + 1 + 1 + 1 + 1 + 1 + 1 + 1 + 1 + 1 + 1 + 1 + 1 = 30 
    \end{align}
  \item[align com quebra de linha] 
    \begin{align}
      1 + 1 + 1 + 1 + 1 + 1 + 1 + 1 + 1 + 1 + 1 + 1 + 1 + 1 + 1 \\
      + 1 + 1 + 1 + 1 + 1 + 1 + 1 + 1 + 1 + 1 + 1 + 1 + 1 + 1 + 1 = 30 
    \end{align}
\end{description}

No caso do desenvolvimento/simplificação de uma expressão
matemática também é recomendado utilizar os ambientes
disponibilizados pelo pacote breqn.
\begin{description}
  \item[dmath]
    \begin{dmath}
      f(x) = 1 + 1 + 1 + 1 + 1 + 1 + 1 + 1 + 1 + 1 + 1 + 1 + 1 + 1 + 1 + 1 + 1 +
      1 + 1 + 1 + 1 + 1 + 1 + 1 + 1 + 1 + 1 + 1 + 1 + 1
      = 2 + 2 + 2 + 2 + 2 + 2 + 2 + 2 + 2 + 2 + 2 + 2 + 2 + 2 + 2
      = 4 + 4 + 4 + 4 + 4 + 4 + 4 + 2
      = 8 + 8 + 8 + 6
      = 16 + 14
      = 30
    \end{dmath}
  \item[equation com split] 
    \begin{equation}
      \begin{split}
        f(x) &= 1 + 1 + 1 + 1 + 1 + 1 + 1 + 1 + 1 + 1 + 1 + 1 + 1 + 1 + 1 + 1 + 1 \\
        &\quad {}+ 1 + 1 + 1 + 1 + 1 + 1 + 1 + 1 + 1 + 1 + 1 + 1 + 1 \\
        &= 2 + 2 + 2 + 2 + 2 + 2 + 2 + 2 + 2 + 2 + 2 + 2 + 2 + 2 + 2 \\
        &= 4 + 4 + 4 + 4 + 4 + 4 + 4 + 2 \\
        &= 8 + 8 + 8 + 6 \\
        &= 16 + 14 \\
        &= 30
      \end{split}
    \end{equation}
  \item[align com quebra de linha] 
    \begin{align}
        f(x) &= 1 + 1 + 1 + 1 + 1 + 1 + 1 + 1 + 1 + 1 + 1 + 1 + 1 + 1 + 1 + 1 + 1 \\
        &\quad {}+ 1 + 1 + 1 + 1 + 1 + 1 + 1 + 1 + 1 + 1 + 1 + 1 + 1 \\
        &= 2 + 2 + 2 + 2 + 2 + 2 + 2 + 2 + 2 + 2 + 2 + 2 + 2 + 2 + 2 \\
        &= 4 + 4 + 4 + 4 + 4 + 4 + 4 + 2 \\
        &= 8 + 8 + 8 + 6 \\
        &= 16 + 14 \\
        &= 30
    \end{align}
\end{description}

Para o caso de equações relacionadas e que devem ser agrupadas,
temos
\begin{description}
  \item[dgroup com dmath]
    \begin{dgroup}
      \begin{dmath}
        f(x) = 1 + 1 + 1 + 1 + 1 + 1 + 1 + 1 + 1 + 1 + 1 + 1 + 1 + 1 + 1 + 1 + 1
        + 1 + 1 + 1 + 1 + 1 + 1 + 1 + 1 + 1 + 1 + 1 + 1 + 1
        = 2 + 2 + 2 + 2 + 2 + 2 + 2 + 2 + 2 + 2 + 2 + 2 + 2 + 2 + 2
        = 4 + 4 + 4 + 4 + 4 + 4 + 4 + 2
        = 8 + 8 + 8 + 6
        = 16 + 14
        = 30
      \end{dmath}
      \begin{dmath}
        g(x) = 2 + 2 + 2 + 2
        = 4 + 4
        = 8
      \end{dmath}
    \end{dgroup}
  \item[subequations com equation com split] 
    \begin{subequations}
      \begin{equation}
        \begin{split}
          f(x) &= 1 + 1 + 1 + 1 + 1 + 1 + 1 + 1 + 1 + 1 + 1 + 1 + 1 + 1 + 1 + 1 + 1 \\
          &\quad {}+ 1 + 1 + 1 + 1 + 1 + 1 + 1 + 1 + 1 + 1 + 1 + 1 + 1 \\
          &= 2 + 2 + 2 + 2 + 2 + 2 + 2 + 2 + 2 + 2 + 2 + 2 + 2 + 2 + 2 \\
          &= 4 + 4 + 4 + 4 + 4 + 4 + 4 + 2 \\
          &= 8 + 8 + 8 + 6 \\
          &= 16 + 14 \\
          &= 30
        \end{split}
      \end{equation}
      \begin{equation}
        \begin{split}
          g(x) &= 2 + 2 + 2 + 2 \\
          &= 4 + 4 \\
          &= 8
        \end{split}
      \end{equation}
    \end{subequations}
  \item[align com split] 
    \begin{align}
      \begin{split}
        f(x) &= 1 + 1 + 1 + 1 + 1 + 1 + 1 + 1 + 1 + 1 + 1 + 1 + 1 + 1 + 1 + 1 + 1 \\
        &\quad {}+ 1 + 1 + 1 + 1 + 1 + 1 + 1 + 1 + 1 + 1 + 1 + 1 + 1 \\
        &= 2 + 2 + 2 + 2 + 2 + 2 + 2 + 2 + 2 + 2 + 2 + 2 + 2 + 2 + 2 \\
        &= 4 + 4 + 4 + 4 + 4 + 4 + 4 + 2 \\
        &= 8 + 8 + 8 + 6 \\
        &= 16 + 14 \\
        &= 30
      \end{split} \\
      \begin{split}
        g(x) &= 2 + 2 + 2 + 2 \\
        &= 4 + 4 \\
        &= 8
      \end{split}
    \end{align}
\end{description}

\subsection{Referência cruzada}
Parte das equações anteriores encontram-se numeradas. Esse número pode ser
facilmente acessado se junto da equação tiver sido utilizando o comando label:
\begin{dmath}
  c^2 = a^2 + b^2. \label{eq:exem_pitagoras}
\end{dmath}
E para acessar o número utiliza o comando eqref, \eqref{eq:exem_pitagoras}.

Além de numerar equações também é possível nomeá-las utilizando o comando
tag\footnote{O pacote breqn não possue suporte ao comando tag.}:
\begin{align}
  c^2 &= a^2 + b^2 - 2 a b \cos\theta 
  \label{eq:exem_pitagoras_generalizado}
  \tag{GTP}
\end{align}
E para acessar o nome utiliza-se o comando eqref,
\eqref{eq:exem_pitagoras_generalizado}.

Para que no pdf não apareça o parâmetro dos comandos label é preciso remover o
pacote showlabels do arquivo pacotes.tex.

\section{Definições}

Vários ambientes já estão definidos como: Teorema, Conjectura, Corolário,
Definição, \ldots

\begin{thm}
Teorema, Teorema, Teorema, Teorema.
\end{thm}

\begin{con}
Conjectura, Conjecture, Conjectura, Conjectura.
\end{con}

\begin{cor}
Corolário, Corolário, Corolário, Corolário.
\end{cor}

\begin{dfn}
Definição, Definição, Definição.
\end{dfn}

Use esses ambientes de maneira sábia.

\chapter{Título 2}
Esse é o segundo capítulo da sua defesa.

\section{Seção 2.1}
Essa é uma seção da sua defesa.

É possível inserir figuras na sua defesa. \\
\includegraphics{figuras/unicamp-logo.jpg}

É recomendado deixar as figuras\index{figura} como objetos flutuantes.
\begin{figure}[!htb]
  \center
  \includegraphics[width=4cm]{figuras/unicamp-logo.jpg}
  \caption{Logo da UNICAMP.}
  \label{fig:log_unicamp}
\end{figure}

Além de inserir figuras é possível ``desenhar''\index{figura!desenhar} figuras
utilizando o pacote TikZ\index{figura!TikZ}, como na Figura~\ref{fig:exem_tikz}.
\begin{figure}[!htb]
  \centering
  \begin{tikzpicture}
    \draw (0,0) -- (0,2) -- (2,2) -- (2,0) -- (0,0);
  \end{tikzpicture}
  \caption{Exemplo do uso do TikZ.}
  \label{fig:exem_tikz}
\end{figure}

\section{Seção 2.2}
Além de figuras também é possível inserir tabelas. \\
\begin{tabular}{cc}
  \toprule
  Tabela & Tabela \\
  \midrule
  Tabela & Tabela \\
  \bottomrule
\end{tabular}

Assim como as figuras, é recomendado que as tabelas sejam incluídas como
objetos flutuantes.
\begin{table}[!htb]
  \caption{Exemplo de Tabela.}
  \label{tab:exem}
  \centering
  \begin{tabular}{cc}
    \toprule
    Tabela & Tabela \\
    \midrule
    Tabela & Tabela \\
    \bottomrule
  \end{tabular}
\end{table}

Além de inserir tabelas manualmente, como na Tabela~\ref{tab:exem}, também é
possível utilizar arquivos \emph{.csv} para criar tabelas. Neste caso, é
necessário incluir o pacote \texttt{csvsimple} que não foi incluído por questão
de compatibilidade com distribuições do \LaTeX \ anteriores a 2009.

\chapter{T\'itulo 3}
Esse \'e o terceiro cap\'itulo da sua tese.

\section{Se\c c\~ao 3.1}
Para facilitar a leitura, é bom incluir o algoritmo de forma adequada e não
como uma lista.
\begin{algorithmic}[2]
    \STATE $i \leftarrow 0$
    \STATE $y \leftarrow 0$
    \FOR{$i \leq 10$}
        \STATE $y \leftarrow y + i$
        \STATE $i \leftarrow i + 1$
    \ENDFOR
\end{algorithmic}

É possível nomear os algoritmos.
\begin{algorithm}
    \caption{Loop infinito.}
    \label{alg:loop_inf}
    \begin{algorithmic}
        \REQUIRE $n \geq 0$
        \ENSURE $y = 1 + 2 + \ldots + n$
        \STATE $y \leftarrow 0$
        \STATE $i \leftarrow 0$
        \IF{$n < 0$}
            \PRINT Entrada inadequada.
        \ELSE
            \WHILE{$i \neq n$}
                \STATE $y \leftarrow y + i$
            \ENDWHILE
        \ENDIF
    \end{algorithmic}
\end{algorithm}

Para algoritmos, recomenda-se utilizar o pacote \emph{algorithmic} pois este
permite a tradução dos ``comandos''.

Em alguns casos, recomenda-se incluir um trecho de código.
\lstinputlisting[firstline=5, lastline=5, nolol=false]{src/exem.c}

Assim como os algoritmos também é possível nomear os códigos.
\lstinputlisting[firstline=10, lastline=12, caption=Loop em C,
language=C,  frame=single]{src/exem.c}

Para trecho de códigos,  recomenda-se utilizar o pacote \emph{listings} pois
este possue várias opções.

% TODO Inserir os arquivos referentes ao corpo da tese.
%
% FIXME Se nao for utilizar apendices ou anexos, comentar a linha abaixo.
\appendix
% FIXME Remover as 2 linhas abaixo.
\chapter{T\'{i}tulo A}
Esse \'{e} um Ap\^{e}ndice da sua tese.

\section{Se\c{c}\~{a}o A.1}
Essa é uma se\c{c}\~{a}o de um dos ap\^{e}ndices da sua tese.

\chapter*{Anexo}
\addcontentsline{toc}{chapter}{Anexo}
Esse é um anexo.

% TODO Inserir os arquivos referentes aos apendices e anexos.
%
\backmatter
% FIXME Remover a linha abaixo.
\nocite{*}
% FIXME Mudar o estilo para o de sua preferencia.
\bibliographystyle{amsplain}
% FIXME Adicionar os arquivos .bib que for utilizar ou editar o arquivo
% tese.bib.
\bibliography{tese}
\addcontentsline{toc}{chapter}{Refer\^{e}ncia Bibliogr\'{a}fica}
%
\clearpage
\addcontentsline{toc}{chapter}{\'{I}ndice Remissivo}
\printindex
\end{document}
