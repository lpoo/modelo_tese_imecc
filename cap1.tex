\chapter{T\'itulo 1}
Esse \'e o primeiro cap\'itulo da sua tese.

\section{Se\c c\~ao 1.1}
Essa \'{e} uma se\c{c}\~{a}o da sua tese.

\subsection{Subse\c{c}\~{a}o 1.1.1}

Essa \'{e} uma subse\c{c}\~{a}o da sua tese.

\subsection{Subse\c{c}\~{a}o 1.1.2}

Essa \'{e} outra subse\c{c}\~{a}o da sua tese.

\section{Equa\c{c}\~{o}es matem\'{a}ticas}
Para equa\c{c}\~{o}es matem\'{a}ticas est\'{a} sendo utilizado os seguintes
pacotes:
\begin{itemize}
  \item amsmath,
  \item amsfonts,
  \item amssymb,
  \item amsthm,
  \item breqn.
\end{itemize}

Um exemplo de equa\c{c}\~{a}o na mesma linha: 
$ 1 + 1 + 1 + 1 + 1 + 1 = 6$, 
que \'{e} trivial\index{trivial} de ser verificada.

Equa\c{c}\~{o}es na mesma linha s\~{a}o quebradas automaticamente:
$ 1 + 1 + 1 + 1 + 1 + 1 + 1 + 1 + 1 + 1 + 1 + 1 = 12$. 

Ao utilizar equa\c{c}\~{o}es na mesma linha deve-se tomar o cuidado de manter a
legibilidade da equa\c{c}\~{a}o e n\~{a}o modificar a altura da linha. \'{E}
errado utilizar $\frac{1}{2} + \frac{1}{2} = 1$ ou $\frac{\partial}{\partial x}
(y^2 + 2xy + x^2) = 2y + 2x$, devendo ser utilizado $(1/2) + (1/2) = 1$ ou
$\partial (y^2 + 2 x y + x^2) / \partial x = 2 y + 2 x$.

Al\'{e}m de equa\c{c}\~{o}es na mesma linha, tamb\'{e}m \'{e} poss\'{i}vel
utilizar equa\c{c}\~{o}es em destaque:
\begin{equation}
1 + 1 + 1 + 1 + 1 + 1 = 6
\end{equation}
ou
\begin{equation*}
1 + 1 + 1 + 1 + 1 + 1 = 6.
\end{equation*}
Ao utilizar equa\c{c}\~{o}es matem\'{a}ticas em destaque, n\~{a}o esque\c{c}a da pontua\c{c}\~{a}o nas
equa\c{c}\~{o}es.

Recomenda-se utilizar os ambientes
\begin{itemize}
  \item dmath(*) e
  \item dgroup(*)
\end{itemize}
disponibilizados pelo pacote breqn ao inv\'{e}s dos ambientes
\begin{itemize}
  \item equation(*,)
  \item align(*),
  \item multiline(*) e
  \item split
\end{itemize}
pois os primeiros quebram e alinham automaticamente as equa\c{c}\~{o}es em destaque.
Veja a seguir a ``equival\^{e}ncia'' entre os ambientes:
\begin{description}
  \item[dmath]
    \begin{dmath}
      1 + 1 + 1 + 1 + 1 + 1 + 1 + 1 + 1 + 1 + 1 + 1 + 1 + 1 + 1 + 1 + 1 + 1 + 1
      + 1 + 1 + 1 + 1 + 1 + 1 + 1 + 1 + 1 + 1 + 1 = 30
    \end{dmath}
  \item[equation] 
    \begin{equation}
      1 + 1 + 1 + 1 + 1 + 1 + 1 + 1 + 1 + 1 + 1 + 1 + 1 + 1 + 1 + 1 + 1 + 1 + 1
      + 1 + 1 + 1 + 1 + 1 + 1 + 1 + 1 + 1 + 1 + 1 = 30
    \end{equation}
  \item[equation com split] 
    \begin{equation}
      \begin{split}
        1 + 1 + 1 + 1 + 1 + 1 + 1 + 1 + 1 + 1 + 1 + 1 + 1 + 1 + 1 \\
        + 1 + 1 + 1 + 1 + 1 + 1 + 1 + 1 + 1 + 1 + 1 + 1 + 1 + 1 + 1 = 30 
      \end{split}
    \end{equation}
  \item[align] 
    \begin{align}
      1 + 1 + 1 + 1 + 1 + 1 + 1 + 1 + 1 + 1 + 1 + 1 + 1 + 1 + 1 \\
      + 1 + 1 + 1 + 1 + 1 + 1 + 1 + 1 + 1 + 1 + 1 + 1 + 1 + 1 + 1 = 30 
    \end{align}
  \item[align com quebra de linha] 
    \begin{align}
      1 + 1 + 1 + 1 + 1 + 1 + 1 + 1 + 1 + 1 + 1 + 1 + 1 + 1 + 1 \\
      + 1 + 1 + 1 + 1 + 1 + 1 + 1 + 1 + 1 + 1 + 1 + 1 + 1 + 1 + 1 = 30 
    \end{align}
\end{description}

No caso do desenvolvimento/simplifica\c{c}\~{a}o de uma express\~{a}o
matem\'{a}tica tamb\'{e}m \'{e} recomendado utilizar os ambientes
disponibilizados pelo pacote breqn.
\begin{description}
  \item[dmath]
    \begin{dmath}
      f(x) = 1 + 1 + 1 + 1 + 1 + 1 + 1 + 1 + 1 + 1 + 1 + 1 + 1 + 1 + 1 + 1 + 1
      + 1 + 1 + 1 + 1 + 1 + 1 + 1 + 1 + 1 + 1 + 1 + 1 + 1
      = 2 + 2 + 2 + 2 + 2 + 2 + 2 + 2 + 2 + 2 + 2 + 2 + 2 + 2 + 2
      = 4 + 4 + 4 + 4 + 4 + 4 + 4 + 2
      = 8 + 8 + 8 + 6
      = 16 + 14
      = 30
    \end{dmath}
  \item[equation com split] 
    \begin{equation}
      \begin{split}
        f(x) &= 1 + 1 + 1 + 1 + 1 + 1 + 1 + 1 + 1 + 1 + 1 + 1 + 1 + 1 + 1 + 1 + 1 \\
        &\quad {}+ 1 + 1 + 1 + 1 + 1 + 1 + 1 + 1 + 1 + 1 + 1 + 1 + 1 \\
        &= 2 + 2 + 2 + 2 + 2 + 2 + 2 + 2 + 2 + 2 + 2 + 2 + 2 + 2 + 2 \\
        &= 4 + 4 + 4 + 4 + 4 + 4 + 4 + 2 \\
        &= 8 + 8 + 8 + 6 \\
        &= 16 + 14 \\
        &= 30
      \end{split}
    \end{equation}
  \item[align com quebra de linha] 
    \begin{align}
        f(x) &= 1 + 1 + 1 + 1 + 1 + 1 + 1 + 1 + 1 + 1 + 1 + 1 + 1 + 1 + 1 + 1 + 1 \\
        &\quad {}+ 1 + 1 + 1 + 1 + 1 + 1 + 1 + 1 + 1 + 1 + 1 + 1 + 1 \\
        &= 2 + 2 + 2 + 2 + 2 + 2 + 2 + 2 + 2 + 2 + 2 + 2 + 2 + 2 + 2 \\
        &= 4 + 4 + 4 + 4 + 4 + 4 + 4 + 2 \\
        &= 8 + 8 + 8 + 6 \\
        &= 16 + 14 \\
        &= 30
    \end{align}
\end{description}

Para o caso de equa\c{c}\~{o}es relacionadas e que devem ser agrupadas,
temos
\begin{description}
  \item[dgroup com dmath]
    \begin{dgroup}
      \begin{dmath}
        f(x) = 1 + 1 + 1 + 1 + 1 + 1 + 1 + 1 + 1 + 1 + 1 + 1 + 1 + 1 + 1 + 1 + 1
        + 1 + 1 + 1 + 1 + 1 + 1 + 1 + 1 + 1 + 1 + 1 + 1 + 1
        = 2 + 2 + 2 + 2 + 2 + 2 + 2 + 2 + 2 + 2 + 2 + 2 + 2 + 2 + 2
        = 4 + 4 + 4 + 4 + 4 + 4 + 4 + 2
        = 8 + 8 + 8 + 6
        = 16 + 14
        = 30
      \end{dmath}
      \begin{dmath}
        g(x) = 2 + 2 + 2 + 2
        = 4 + 4
        = 8
      \end{dmath}
    \end{dgroup}
  \item[subequations com equation com split] 
    \begin{subequations}
      \begin{equation}
        \begin{split}
          f(x) &= 1 + 1 + 1 + 1 + 1 + 1 + 1 + 1 + 1 + 1 + 1 + 1 + 1 + 1 + 1 + 1 + 1 \\
          &\quad {}+ 1 + 1 + 1 + 1 + 1 + 1 + 1 + 1 + 1 + 1 + 1 + 1 + 1 \\
          &= 2 + 2 + 2 + 2 + 2 + 2 + 2 + 2 + 2 + 2 + 2 + 2 + 2 + 2 + 2 \\
          &= 4 + 4 + 4 + 4 + 4 + 4 + 4 + 2 \\
          &= 8 + 8 + 8 + 6 \\
          &= 16 + 14 \\
          &= 30
        \end{split}
      \end{equation}
      \begin{equation}
        \begin{split}
          g(x) &= 2 + 2 + 2 + 2 \\
          &= 4 + 4 \\
          &= 8
        \end{split}
      \end{equation}
    \end{subequations}
  \item[align com split] 
    \begin{align}
      \begin{split}
        f(x) &= 1 + 1 + 1 + 1 + 1 + 1 + 1 + 1 + 1 + 1 + 1 + 1 + 1 + 1 + 1 + 1 + 1 \\
        &\quad {}+ 1 + 1 + 1 + 1 + 1 + 1 + 1 + 1 + 1 + 1 + 1 + 1 + 1 \\
        &= 2 + 2 + 2 + 2 + 2 + 2 + 2 + 2 + 2 + 2 + 2 + 2 + 2 + 2 + 2 \\
        &= 4 + 4 + 4 + 4 + 4 + 4 + 4 + 2 \\
        &= 8 + 8 + 8 + 6 \\
        &= 16 + 14 \\
        &= 30
      \end{split} \\
      \begin{split}
        g(x) &= 2 + 2 + 2 + 2 \\
        &= 4 + 4 \\
        &= 8
      \end{split}
    \end{align}
\end{description}

\subsection{Refer\^{e}ncia cruzada}
Al\'{e}m de numerar equa\c{c}\~{o}es, e.g., \eqref{eq:exem_unidade}, tamb\'{e}m \'{e}
poss\'ivel nome\'a-las:
\begin{align}
    g(x) &= 0.
    \tag{EIN}
    \label{eq:exem_zero}
\end{align}

Muito cuidado com o uso da refer\^{e}ncia cruzada, e.g.,
\eqref{eq:exem_unidade} e \eqref{eq:exem_zero}.

V\'arios ambientes j\'a est\~{a}o definidos como: Teorema, Conjectura, Corol\'ario,
Defini\c{c}\~{a}o, \ldots

\begin{thm}
Teorema, Teorema, Teorema, Teorema.
\end{thm}

\begin{con}
Conjectura, Conjecture, Conjectura, Conjectura.
\end{con}

\begin{cor}
Corol\'ario, Corol\'ario, Corol\'ario, Corol\'ario.
\end{cor}

\begin{dfn}
Defini\c{c}\~{a}o, Defini\c{c}\~{a}o, Defini\c{c}\~{a}o.
\end{dfn}

Use esses ambientes de maneira s\'abia.
