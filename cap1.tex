\chapter{T\'itulo 1}
Esse \'e o primeiro cap\'itulo da sua tese.

\section{Se\c c\~ao 1.1}
Essa \'{e} uma se\c{c}\~{a}o da sua tese.

Um exemplo de equa\c{c}\~{a}o na mesma linha: 
$ 1 + 1 + 1 + 1 + 1 + 1 = 6$, 
que \'{e} trivial\index{trivial} de ser verificada.

Um exemplo de equa\c{c}\~{a}o em destaque:
\begin{align*}
1 + 1 + 1 + 1 + 1 + 1 &= 6,
\end{align*}
n\~{a}o esque\c{c}a da pontua\c{c}\~{a}o nas equa\c{c}\~{o}es.

Mais exemplos de equa\c{c}\~{o}es em destaque:
\begin{align*}
1 + 1 + 1 + 1 + 1 + 1 &= 6, \\
2 + 1 + 1 + 1 + 1 &= 6, \\
3 + 1 + 1 + 1 &= 6.
\end{align*}
Novamente n\~{a}o esque\c{c}a da pontua\c{c}\~{a}o.

Para simplifica\c{c}\~{o}es de uma express\~{a}o, voc\^{e} pode utilizar:
\begin{align*}
    f(x) &= 1 + 1 + 1 + 1 + 1 + 1 \\
    &= 2 + 2 + 2 \\
    &= 6.
\end{align*}
Muito cuidado com a pontua\c{c}\~{a}o.

\subsection{Subse\c{c}\~ao 1.1.1}
Essa \'{e} uma subse\c{c}\~{a}o da sua tese.

Para equa\c{c}\~{o}es muito longas, voc\^{e} ter\'a que dividi-la em v\'arias linhas:
\begin{align*}
    f(x) &= 1 + 1 + 1 + 1 + 1 + 1 + 1 + 1 + 1 + 1 \\
    &\quad {}+ 1 + 1 + 1 + 1 + 1 + 1 + 1 + 1 + 1 \\
    &\quad {}+ 1 + 1 + 1 + 1 + 1 + 1 + 1 + 1.
\end{align*}
Existe outras conven\c{c}\~{o}es para dividir equa\c{c}\~{o}es muito longas mas gosto da
mostrada logo acima.

Tamb\'{e}m \'{e} poss\'ivel numerar equa\c{c}\~{o}es:
\begin{align}
    f(x) &= 1.
    \label{eq:exem_unidade}
\end{align}

Al\'{e}m de numerar equa\c{c}\~{o}es, e.g., \eqref{eq:exem_unidade}, tamb\'{e}m \'{e}
poss\'ivel nome\'a-las:
\begin{align}
    g(x) &= 0.
    \tag{EIN}
    \label{eq:exem_zero}
\end{align}

Muito cuidado com o uso da refer\^{e}ncia cruzada, e.g.,
\eqref{eq:exem_unidade} e \eqref{eq:exem_zero}.

\section{T\'itulo 1.2}
Essa \'{e} outra se\c{c}\~{a}o da sua tese.

V\'arios ambientes j\'a est\~{a}o definidos como: Teorema, Conjectura, Corol\'ario,
Defini\c{c}\~{a}o, \ldots

\begin{thm}
Teorema, Teorema, Teorema, Teorema.
\end{thm}

\begin{con}
Conjectura, Conjecture, Conjectura, Conjectura.
\end{con}

\begin{cor}
Corol\'ario, Corol\'ario, Corol\'ario, Corol\'ario.
\end{cor}

\begin{dfn}
Defini\c{c}\~{a}o, Defini\c{c}\~{a}o, Defini\c{c}\~{a}o.
\end{dfn}

Use esses ambientes de maneira s\'abia.
