\chapter{Título 2}
Esse é o segundo capítulo da sua defesa.

\section{Seção 2.1}
Essa é uma seção da sua defesa.

É possível inserir figuras na sua defesa. \\
\includegraphics{figuras/unicamp-logo.jpg}

É recomendado deixar as figuras\index{figura} como objetos flutuantes.
\begin{figure}[!htb]
  \center
  \includegraphics[width=4cm]{figuras/unicamp-logo.jpg}
  \caption{Logo da UNICAMP.}
  \label{fig:log_unicamp}
\end{figure}

Além de inserir figuras é possível ``desenhar''\index{figura!desenhar} figuras
utilizando o pacote TikZ\index{figura!TikZ}, como na Figura~\ref{fig:exem_tikz}.
\begin{figure}[!htb]
  \centering
  \begin{tikzpicture}
    \draw (0,0) -- (0,2) -- (2,2) -- (2,0) -- (0,0);
  \end{tikzpicture}
  \caption{Exemplo do uso do TikZ.}
  \label{fig:exem_tikz}
\end{figure}

\section{Seção 2.2}
Além de figuras também é possível inserir tabelas. \\
\begin{tabular}{cc}
  \toprule
  Tabela & Tabela \\
  \midrule
  Tabela & Tabela \\
  \bottomrule
\end{tabular}

Assim como as figuras, é recomendado que as tabelas sejam incluídas como
objetos flutuantes.
\begin{table}[!htb]
  \caption{Exemplo de Tabela.}
  \label{tab:exem}
  \centering
  \begin{tabular}{cc}
    \toprule
    Tabela & Tabela \\
    \midrule
    Tabela & Tabela \\
    \bottomrule
  \end{tabular}
\end{table}

Além de inserir tabelas manualmente, como na Tabela~\ref{tab:exem}, também é
possível utilizar arquivos \emph{.csv} para criar tabelas. Neste caso, é
necessário incluir o pacote \texttt{csvsimple} que não foi incluído por questão
de compatibilidade com distribuições do \LaTeX \ anteriores a 2009.
