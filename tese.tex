% FIXME Ao final, deixe descomentada a linha correspondente ao numero de paginas
% que a sua defesa possui.
\documentclass[12pt, letterpaper, oneside]{book}  % Para menos de 100 paginas.
% \documentclass[12pt, letterpaper, twoside]{book}  % Para mais de 100 paginas.

%%%%%%%%%%%%%%%%%%%%%%%%%%%% Configuração: pacotes %%%%%%%%%%%%%%%%%%%%%%%%%%%%
%%%%%%%%%%%%%%%%%%%%%%%%%%%%%% Pacotes: básicos %%%%%%%%%%%%%%%%%%%%%%%%%%%%%% 
\usepackage[utf8]{inputenc}
\usepackage{cmap}
\usepackage[T1]{fontenc}
\usepackage[brazilian]{babel}
\usepackage{indentfirst}
\usepackage[top=3cm,bottom=3cm,right=2cm,left=2cm]{geometry}


%%%%%%%%%%%%%%%%%%%%%%%%%%%%%%% Pacotes: links %%%%%%%%%%%%%%%%%%%%%%%%%%%%%%%
\usepackage[hyphens]{url}
\usepackage{breakurl}
\usepackage{hyperref}
% FIXME Comente o pacote abaixo quando for concluir sua defesa e for entregar a
% versão final.
\usepackage{showkeys}


%%%%%%%%%%%%%%%%%%%%%%%%%%%%%%%% Pacotes: ams %%%%%%%%%%%%%%%%%%%%%%%%%%%%%%%% 
\usepackage{amsmath}
\usepackage{amsfonts}
\usepackage{amssymb}
\usepackage{amsthm}
\usepackage{breqn}


%%%%%%%%%%%%%%%%%%%%%%%%%%%%%% Pacotes: tabelas %%%%%%%%%%%%%%%%%%%%%%%%%%%%%%
\usepackage{multicol}
\usepackage{multirow}
\usepackage{array}
\usepackage{booktabs}


%%%%%%%%%%%%%%%%%%%%%%%%%%%%%% Pacotes: figuras %%%%%%%%%%%%%%%%%%%%%%%%%%%%%% 
\usepackage{pdfpages}
\usepackage{graphicx}
\usepackage{tikz}
\usepackage{wrapfig}


%%%%%%%%%%%%%%%%%%%%%%%%%%%%% Pacotes: algoritmos %%%%%%%%%%%%%%%%%%%%%%%%%%%%% 
\usepackage{algorithmic}
\usepackage[chapter]{algorithm}
\floatname{algorithm}{Algoritmo}
\renewcommand{\listalgorithmname}{Lista de Algoritmos}


%%%%%%%%%%%%%%%%%%%%%%%%%%%%%% Pacotes: códigos %%%%%%%%%%%%%%%%%%%%%%%%%%%%%% 
\usepackage{textcomp}
\usepackage{listings}
\renewcommand\lstlistingname{Código}
\renewcommand\lstlistlistingname{Lista de Códigos}


%%%%%%%%%%%%%%%%%%%%%%%%%%%%%%% Pacotes: index %%%%%%%%%%%%%%%%%%%%%%%%%%%%%%% 
\usepackage{makeidx}
\makeindex


%%%%%%%%%%%%%%%%%%%%%%%%%%%%%%% Pacotes: fontes %%%%%%%%%%%%%%%%%%%%%%%%%%%%%% 
\usepackage{lmodern}
\usepackage{mathrsfs}

%%%%%%%%%%%%%%%%%%%%%%%%%%%%%%% Pacotes: unidades %%%%%%%%%%%%%%%%%%%%%%%%%%%% 
\usepackage{siunitx}

% TODO Inserir pacotes adicionais aqui.
  % Arquivo com os pacotes.

%%%%%%%%%%%%%%%%%%%%%%%%% Configuração: dados pessoais %%%%%%%%%%%%%%%%%%%%%%%%%
% FIXME Substituir 'Nome completo do aluno' pelo seu nome.
\newcommand{\autor}{Nome completo do aluno}
% FIXME Se for do sexo feminino, descomente a linha a seguir.
% \def\femaleAuthor{}

% FIXME Substituir 'Título da defesa' pelo título da defesa.
\newcommand{\titulo}{Título da defesa}
% FIXME Se estiver no programa de mestrado, descomente a linha a seguir.
% \def\mestrado{}
% FIXME Deixe descomente apenas a linha referente ao departamento.
% \def\matematica{}
\def\aplicada{}
% \def\estatistica{}

% FIXME Substituir 'Nome completo do orientador' pelo nome completo do seu
% orientador.
\newcommand{\orientador}{Nome completo do orientador}
% FIXME Se for orientado por uma mulher, descomente a linha a seguir.
% \def\femaleOrientador{}

% FIXME Substituir 'Nome completo do coorientador' pelo nome completo do seu
% coorientador. Caso não tenha coorientador, comente a linha a seguir.
\newcommand{\coorientador}{Nome completo do coorientador}
% FIXME Se for coorientado por uma mulher, descomente a linha a seguir.
% \def\femaleCoorientador{}

% FIXME Substituir 'Ano' pelo ano em que ocorreu sua defesa.
\newcommand{\ano}{Ano}

%%%%%%%%%%%%%%%%%%%%%%%%%%% Configuração: definições %%%%%%%%%%%%%%%%%%%%%%%%%%%
%%%%%%%%%%%%%%%%%%%%%%%%%%%%% Configurações: links %%%%%%%%%%%%%%%%%%%%%%%%%%%%%
\hypersetup{
% TODO Por padrão os links, no pdf, para equações, figuras, referencias,
% tabelas, urls são identificados por uma caixa colorida em volta do link. Essa
% caixa colorida não eh impressa mas pode atrapalhar a leitura para alguns. Se
% desejar removê-las descomente a linha abaixo.
% hidelinks,
pdftitle={\titulo},  % Não modifique esta linha.
pdfauthor={\autor}  % Não modifique esta linha.
}


%%%%%%%%%%%%%%%%%%%%%%%%%% Configurações: numeração %%%%%%%%%%%%%%%%%%%%%%%%%% 
\numberwithin{equation}{section}
\numberwithin{section}{chapter}


%%%%%%%%%%%%%%%%%%%%%%%%%%%%% Configurações: ams %%%%%%%%%%%%%%%%%%%%%%%%%%%%% 
\theoremstyle{definition}
\newtheorem{thm}{Teorema}[section]
\newtheorem{con}[thm]{Conjectura}
\newtheorem{cor}[thm]{Corolário}
\newtheorem{dfn}[thm]{Definição}
\newtheorem{exm}[thm]{Exemplo}
\newtheorem{lem}[thm]{Lema}
\newtheorem{obs}[thm]{Observação}
\newtheorem{pps}[thm]{Proposição}


% TODO Inserir configurações adicionais aqui.
  % Arquivo com algumas configurações.

%%%%%%%%%%%%%%%%%%%%%%%%%% Início do texto da defesa %%%%%%%%%%%%%%%%%%%%%%%%%%%
\begin{document}
% WARNING Todas as paginas deverão ser numeradas.
%
% As paginas iniciais são numeradas com algoritmos romanos em sua forma
% minuscula.
\frontmatter
%
\thispagestyle{plain}
\includegraphics[width=.94in, height=1in,
keepaspectratio=true]{figuras/unicamp-logo.jpg}
\vspace*{1cm}
\begin{center}
  % O tamanho da fonte deve ser 16pt.
  % Deve-se utilizar caixa alta.
  {\Large \scshape \autor}
\end{center}
\vspace{4cm}
\begin{center}
  % O tamanho da fonte deve ser 16pt em negrito.
  % Deve-se utilizar caixa alta.
  {\Large \scshape \bfseries \titulo}
\end{center}
\vfill
\begin{center}
  % O tamanho da fonte deve ser 12pt em negrito.
  % Deve-se utilizar caixa alta.
  {\bfseries CAMPINAS \\ \ano}
\end{center}
  % Não edite esse arquivo.
\newpage\mbox{}\thispagestyle{plain}\newpage  % Pagina em branco.
%
% WARNING A folha de rosto precisa ser assinada pelos orientadores.
% FIXME Substitua arquivo folha-de-rosto.pdf por uma copia escaneada, comente
% esta linha e descomente a próxima.
\thispagestyle{plain}
% WARNING Não modifique este arquivo.
\includegraphics[width=.94in, height=1in,
keepaspectratio=true]{figuras/unicamp-logo.jpg}
\begin{center}
  {\large \scshape \bfseries Universidade Estadual de Campinas
  \vspace{.5cm}

  Instituto de Matemática, Estatística \\
  e Computação Científica}
\end{center}
\vspace{.7cm}
\begin{center}
  {\large \scshape \bfseries \autor}
\end{center}
\vspace{.7cm}
\begin{center}
  {\Large \scshape \bfseries \titulo}
\end{center}
\vspace{.8cm}
{\bfseries
\noindent
Orientador\ifx\femaleOrientador\undefined
\else
a\fi: Prof\ifx\femaleOrientador\undefined
\else
a\fi. Dr\ifx\femaleOrientador\undefined
\else
a\fi. \orientador
\vspace{.25cm}

\ifx\coorientador\undefined
\else
\noindent
Coorientador\ifx\femaleCoorientador\undefined
\else
a\fi: Prof\ifx\femaleCoorientador\undefined
\else
a\fi. Dr\ifx\femaleCoorientador\undefined
\else
a\fi. \coorientador
\fi
}

\vspace{.5cm}
\begin{flushright}
  \begin{minipage}[c]{.8\textwidth}
    \begin{flushright}
      \ifx\mestrado\undefined
      Tese de doutorado
      \else
      Dissertação de mestrado
      \fi
      apresentada do Instituto de \\ Matemática,
      Estatística e Computação Científica
      da Unicamp para \\ obtenção do título de
      \ifx\mestrado\undefined
      \ifx\femaleAuthor\undefined
      Doutor
      \else
      Doutora
      \fi
      \else
      \ifx\femaleAuthor\undefined
      Mestre
      \else
      Mestra
      \fi
      \fi
      em
      \ifx\matematica\undefined
      \else
      matemática.
      \fi
      \ifx\aplicada\undefined
      \else
      matemática aplicada.
      \fi
      \ifx\estatistica\undefined
      \else
      estatística.
      \fi
    \end{flushright}
  \end{minipage}
\end{flushright}
\vspace{.5cm}
\noindent
{\footnotesize \scshape
Este exemplar corresponde à versão final da 
\ifx\mestrado\undefined
tese
\else
dissertação
\fi \\
defendida 
\ifx\femaleAuthor\undefined
pelo aluno
\else
pela aluna
\fi
\autor,\\
e orientada pel\ifx\femaleOrientador\undefined
o\else
a\fi Prof\ifx\femaleOrientador\undefined
\else
a\fi. Dr\ifx\femaleOrientador\undefined
\else
a\fi. \orientador.
}
\vspace{1cm}

\noindent
{\small \bfseries
\noindent
Assinatura
\ifx\femaleOrientador\undefined
do Orientador
\else
da Orientadora
\fi

\vspace{.7cm}
\noindent
\rule[1pt]{7cm}{.5pt}  % Linha para assinatura do orientador
}
\vspace{.5cm}


\ifx\coorientador\undefined
\else
{\small \bfseries
\noindent
Assinatura
\ifx\femaleCoorientador\undefined
do Coorientador
\else
da Coorientadora
\fi

\vspace{.7cm}
\noindent
\rule[1pt]{7cm}{.5pt}  % Linha para assinatura do coorientador
}
\fi
\vfill
\begin{center}
  {\small \scshape \bfseries Campinas \\ \ano}
\end{center}

% \includepdf{folha-de-rosto.pdf}
%
% WARNING A ficha catalográfica deve estar no verso da folha de rosto.
% FIXME O arquivo ficha-catalografica.pdf deve ser sobrescrito com uma cópia
% do arquivo pdf que a biblioteca lhe enviar.
\includepdf{ficha-catalografica}
%
% WARNING A folha de aprovação deve ser assinada pelos membros da banca apos a
% defesa.
% FIXME Substitua o arquivo folha-de-aprovacao.pdf por uma copia escaneada.
\includepdf{folha-de-aprovacao}
%
% FIXME Se não for incluir a dedicatória, comentar a linha abaixo.
\chapter*{}
\begin{center}
  \emph{
  % FIXME Remover as duas linhas abaixo.
  Aos meus \ldots. \\
  Para \ldots.} 
  % TODO Inserir a dedicatória aqui.
\end{center}

%
% FIXME Se não for incluir a epigrafe, comentar a linha abaixo.
%%TODO Escrever o modelo da epigrafe.
\chapter*{}
\vfill
\begin{flushright}
  Epígrafe \\
  Epígrafe \\
  Epígrafe \\
  Epígrafe
\end{flushright}

%
% FIXME Se não for incluir os agradecimentos, comentar a linha abaixo.
\chapter*{Agradecimentos}
% FIXME Escrever os agradecimentos.
Aqui deve-se agradecer quem merece agradecimento

Se recebeu bolsa de algum órgão de fomento, não esqueça de agradecê-lo.

%
\chapter{Resumo}
% FIXME Remover as 3 linhas abaixo.
resumo resumo resumo
resumo resumo resumo
resumo resumo resumo
% WARNING O resumo deve conter no máximo 500 palavras.
% TODO Inserir o resumo em português aqui.

\vspace{.5cm}
\textbf{Palavras-chave}:
% FIXME Remover a linha abaixo.
palavra01, palavra02, palavra03.
% TODO Inserir as palavras-chave aqui.

\chapter{Abstract}
% FIXME Remover as 3 linhas abaixo.
abstract abstract abstract
abstract abstract abstract
abstract abstract abstract
% WARNING O abstract deve conter no máximo 500 palavras.
% TODO Inserir o resumo em inglês aqui.

\vspace{.5cm}
\textbf{Keywords}:
% FIXME Remover a linha abaixo.
keyword01, keyword02, keyword03.
% TODO Inserir as palavras-chave em inglês aqui.

%
\tableofcontents
% FIXME Comentar as linhas abaixo de não desejar listar as figuras
% apresentadas.
\listoffigures
\addcontentsline{toc}{chapter}{Lista de Figuras}
%
% FIXME Comentar as linhas abaixo de não desejar listar as tabelas
% apresentadas.
\listoftables
\addcontentsline{toc}{chapter}{Lista de Tabelas}
%
% FIXME Comentar as linhas abaixo de não desejar listar os
% algoritmos apresentados.
\listofalgorithms
\addcontentsline{toc}{chapter}{Lista de Algoritmos}
%
% FIXME Comentar as linhas abaixo de não desejar listar os trechos de código
% apresentados.
\lstlistoflistings
\addcontentsline{toc}{chapter}{Lista de Códigos}
%
% FIXME Comentar a linha abaixo de não for apresentar as
% abreviações utilizadas.
\chapter*{Abreviações}
\addcontentsline{toc}{chapter}{Abreviações}

\begin{description}
  % FIXME Remover as duas abreviações abaixo e incluir as que serão
  % utilizadas.
  \item[FIXME] Indica que algo deve ser consertado.
  \item[TODO] Indica que algo deve ser feito.
\end{description}

%
% FIXME Comentar a linha abaixo de não for apresentar os
% símbolos utilizados.
\chapter*{Símbolos}
\addcontentsline{toc}{chapter}{Símbolos}

\begin{description}
    % FIXME Remover os dois símbolos abaixo e incluir as que serão
    % utilizadas.
    \item[$\diamond$] Representa um diamante.
    \item[$\lhd$] Representa um deslocamento para a esquerda.
    \item[$\rhd$] Representa um deslocamento para a direita.
\end{description}

%
% FIXME Comentar a linha abaixo de não for apresentar o
% glossário.
\chapter{Glossário}

\begin{description}
  % FIXME Remover as duas explicações abaixo e incluir as que serão
  % utilizadas.
  \item[Tese] Proposição que é enunciada e sustentada.
  \item[Dissertação] Discurso, exposição ou exame minucioso de
    determinado assunto.
\end{description}

%
% As paginas com o conteúdo da tese são numeradas com algoritmos arábicos.
\mainmatter
%
% FIXME Remover as 3 linhas abaixo.
\chapter{T\'itulo 1}
Esse \'e o primeiro cap\'itulo da sua tese.

\section{Se\c c\~ao 1.1}
Essa \'{e} uma se\c{c}\~{a}o da sua tese.

Um exemplo de equa\c{c}\~{a}o na mesma linha: 
$ 1 + 1 + 1 + 1 + 1 + 1 = 6$, 
que \'{e} trivial\index{trivial} de ser verificada.

Um exemplo de equa\c{c}\~{a}o em destaque:
\begin{align*}
1 + 1 + 1 + 1 + 1 + 1 &= 6,
\end{align*}
n\~{a}o esque\c{c}a da pontua\c{c}\~{a}o nas equa\c{c}\~{o}es.

Mais exemplos de equa\c{c}\~{o}es em destaque:
\begin{align*}
1 + 1 + 1 + 1 + 1 + 1 &= 6, \\
2 + 1 + 1 + 1 + 1 &= 6, \\
3 + 1 + 1 + 1 &= 6.
\end{align*}
Novamente n\~{a}o esque\c{c}a da pontua\c{c}\~{a}o.

Para simplifica\c{c}\~{o}es de uma express\~{a}o, voc\^{e} pode utilizar:
\begin{align*}
    f(x) &= 1 + 1 + 1 + 1 + 1 + 1 \\
    &= 2 + 2 + 2 \\
    &= 6.
\end{align*}
Muito cuidado com a pontua\c{c}\~{a}o.

\subsection{Subse\c{c}\~ao 1.1.1}
Essa \'{e} uma subse\c{c}\~{a}o da sua tese.

Para equa\c{c}\~{o}es muito longas, voc\^{e} terá que dividi-la em várias linhas:
\begin{align*}
    f(x) &= 1 + 1 + 1 + 1 + 1 + 1 + 1 + 1 + 1 + 1 \\
    &\quad {}+ 1 + 1 + 1 + 1 + 1 + 1 + 1 + 1 + 1 \\
    &\quad {}+ 1 + 1 + 1 + 1 + 1 + 1 + 1 + 1.
\end{align*}
Existe outras conven\c{c}\~{o}es para dividir equa\c{c}\~{o}es muito longas mas gosto da
mostrada logo acima.

Tamb\'{e}m \'{e} possível numerar equa\c{c}\~{o}es:
\begin{align}
    f(x) &= 1.
    \label{eq:exem_unidade}
\end{align}

Al\'{e}m de numerar equa\c{c}\~{o}es, e.g., \eqref{eq:exem_unidade}, tamb\'{e}m \'{e}
possível nome\'a-las:
\begin{align}
    g(x) &= 0.
    \tag{EIN}
    \label{eq:exem_zero}
\end{align}

Muito cuidado com o uso da refer\^{e}ncia cruzada, e.g.,
\eqref{eq:exem_unidade} e \eqref{eq:exem_zero}.

\section{Título 1.2}
Essa \'{e} outra se\c{c}\~{a}o da sua tese.

Vários ambientes já est\~{a}o definidos como: Teorema, Conjectura, Corolário,
Defini\c{c}\~{a}o, \ldots

\begin{thm}
Teorema, Teorema, Teorema, Teorema.
\end{thm}

\begin{con}
Conjectura, Conjecture, Conjectura, Conjectura.
\end{con}

\begin{cor}
Corolário, Corolário, Corolário, Corolário.
\end{cor}

\begin{dfn}
Defini\c{c}\~{a}o, Defini\c{c}\~{a}o, Defini\c{c}\~{a}o.
\end{dfn}

Use esses ambientes de maneira sábia.

\chapter{Título 2}
Esse é o segundo capítulo da sua defesa.

\section{Seção 2.1}
Essa é uma seção da sua defesa.

É possível inserir figuras na sua defesa. \\
\includegraphics{figuras/unicamp-logo.jpg}

É recomendado deixar as figuras\index{figura} como objetos flutuantes.
\begin{figure}[!htb]
  \center
  \includegraphics[width=4cm]{figuras/unicamp-logo.jpg}
  \caption{Logo da UNICAMP.}
  \label{fig:log_unicamp}
\end{figure}

Figuras também podem ter subfiguras, usando o pacote \texttt{subcaption}, que 
podem ser citadas individualmente: Figura~\ref{fig:caption1}.
\begin{figure}[!htb]
  \centering
  \subcaptionbox{\emph{Caption} 1.\label{fig:caption1}}
  {\includegraphics[width=0.45\textwidth]{figuras/unicamp-logo.jpg}}
  \qquad
  \subcaptionbox{\emph{Caption} 1.\label{fig:caption2}}
  {\includegraphics[width=0.45\textwidth]{figuras/unicamp-logo.jpg}}
  \caption{Figura com subfiguras.}
  \label{fig:subfig}
\end{figure}

Além de inserir figuras é possível ``desenhar''\index{figura!desenhar} figuras
utilizando o pacote TikZ\index{figura!TikZ}, como na Figura~\ref{fig:exem_tikz}.
\begin{figure}[!htb]
  \centering
  \begin{tikzpicture}
    \draw (0,0) -- (0,2) -- (2,2) -- (2,0) -- (0,0);
  \end{tikzpicture}
  \caption{Exemplo do uso do TikZ.}
  \label{fig:exem_tikz}
\end{figure}

\section{Seção 2.2}
Além de figuras também é possível inserir tabelas. \\
\begin{tabular}{cc}
  \toprule
  Tabela & Tabela \\
  \midrule
  Tabela & Tabela \\
  \bottomrule
\end{tabular}

Assim como as figuras, é recomendado que as tabelas sejam incluídas como
objetos flutuantes.
\begin{table}[!htb]
  \caption{Exemplo de Tabela.}
  \label{tab:exem}
  \centering
  \begin{tabular}{cc}
    \toprule
    Tabela & Tabela \\
    \midrule
    Tabela & Tabela \\
    \bottomrule
  \end{tabular}
\end{table}

Além de inserir tabelas manualmente, como na Tabela~\ref{tab:exem}, também é
possível utilizar arquivos \emph{.csv} para criar tabelas. Neste caso, é
necessário incluir o pacote \texttt{csvsimple} que não foi incluído por questão
de compatibilidade com distribuições do \LaTeX \ anteriores a 2009.

\chapter{Título 3}
Esse é o terceiro capítulo da sua tese.

\section{Seção 3.1}
Para facilitar a leitura, é bom incluir o algoritmo de forma adequada e não
como uma lista.
\begin{algorithmic}[2]
  \STATE $i \leftarrow 0$
  \STATE $y \leftarrow 0$
  \FOR{$i \leq 10$}
    \STATE $y \leftarrow y + i$
    \STATE $i \leftarrow i + 1$
  \ENDFOR
\end{algorithmic}

É possível nomear os algoritmos\index{algoritmo}.
\begin{algorithm}
  \caption{Loop infinito.}
  \label{alg:loop_inf}
  \begin{algorithmic}
    \REQUIRE $n \geq 0$
    \ENSURE $y = 1 + 2 + \ldots + n$
    \STATE $y \leftarrow 0$
    \STATE $i \leftarrow 0$
    \IF{$n < 0$}
      \PRINT Entrada inadequada.
    \ELSE
      \WHILE{$i \neq n$}
        \STATE $y \leftarrow y + i$
      \ENDWHILE
    \ENDIF
  \end{algorithmic}
\end{algorithm}

Para algoritmos, recomenda-se utilizar o pacote
\emph{algorithmic}\index{algoritmo!algorithmic@\emph{algorithmic}} pois este
permite a traduçãoo dos ``comandos''.

Em alguns casos, recomenda-se incluir um trecho de
código\index{codigo@código}.
\lstinputlisting[firstline=5, lastline=5, nolol=true]{src/exem.c}

Assim como os algoritmos também é possível nomear os códigos.
\lstinputlisting[firstline=10, lastline=12, caption=Loop em C,
language=C,  frame=single]{src/exem.c}

Para trecho de códigos,  recomenda-se utilizar o pacote
\emph{listings}\index{codigo@código!listings@\emph{listings}} pois
este possui várias opções.

% TODO Inserir os arquivos referentes ao corpo da tese.
%
% FIXME Se não for utilizar apêndices ou anexos, comentar a linha abaixo.
\appendix
% FIXME Remover as 2 linhas abaixo.
\chapter{Título A}
Esse é um Apêndice da sua tese.

\section{Seção A.1}
Essa é uma seção de um dos apêndices da sua tese.

\chapter*{Anexo}
\addcontentsline{toc}{chapter}{Anexo}
Esse é um anexo.

% TODO Inserir os arquivos referentes aos apêndices e anexos.
%
\backmatter
% FIXME Remover a linha abaixo.
\nocite{*}
% FIXME Mudar o estilo para o de sua preferencia.
\bibliographystyle{amsplain}
% FIXME Adicionar os arquivos .bib que for utilizar ou editar o arquivo
% tese.bib.
\bibliography{tese}
\addcontentsline{toc}{chapter}{Referência Bibliográfica}
%
\clearpage
\addcontentsline{toc}{chapter}{Índice Remissivo}
\printindex
\end{document}
